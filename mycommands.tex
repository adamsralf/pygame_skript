\newref{def}{name=Definition~}
\newref{pri}{name=Prinzip~}
\newref{tab}{name=Tabelle~,names=Tabellen~}
\newref{abb}{name=Abbildung~}
\newref{abschnitt}{name=Abschnitt~}
\newref{kap}{name=Kapitel~}
\newref{zei}{name=Zeile~,names=Zeilen~}
\newref{src}{name=Quelltext~,names=Quelltexte~}
\newref{req}{name=Requirement~}



%Abkürzungen
\newcommand{\true}{\texttt{True}}
\newcommand{\false}{\texttt{False}}
\newcommand{\forSchleife}{\texttt{for}-Schleife}

\newcommand{\myindex}[2]{\setindex{#1}\index{#2}\setindex{main}}

\newcommand{\randnotiz}[1]{%
  \marginpar{%
    \begin{tiny}%
%    \fbox{%
      \parbox{1.5cm}{\color{YellowOrange}#1}%
%    }
    \end{tiny}%
  }%
}

%Requirement
\floatstyle{ruled}
\newfloat{Requirement}{H}{lor}
\newcommand{\br}[2]%
        {\begin{Requirement}%
                        \caption[#1]{\parbox[c][1.5ex][c]{7cm}{#1\color{White}gD}}\label{#2}\index{#1}%
                        \begin{quote}%
                                \begin{itshape}%
                                \vspace{-0.7em}
        }
\newcommand{\er}%
        {%
                                \end{itshape}%
                        \end{quote}%
                        \vspace{-1.5ex}
        \end{Requirement}%
        \vspace{-1.5ex}
        }



%Definition
\floatstyle{ruled}
\newfloat{Definition}{H}{lod}
\newcommand{\bd}[2]%
        {\begin{Definition}%
                        \caption[#1]{\parbox[c][1.5ex][c]{7cm}{#1\color{White}gD}}\label{#2}\index{#1}%
                        \begin{quote}%
                                \begin{itshape}%
                                \vspace{0.4em}
        }
\newcommand{\ed}%
        {%
                                \end{itshape}%
                        \end{quote}%
                        \vspace{-1.5ex}
        \end{Definition}%
        \vspace{-1.5ex}
        }
%Source Code Listen
\floatstyle{ruled}
\newfloat{Quelltext}{H}{losc}

%% lstSource ist die neue Version, sollte immer verwendet werden!
%#1 Dateiname
%#2 Startzeile
%#3 Endezeile
%#4 Sprache
%#5 Überschrift
%#6 Label
\newcounter{StartZeilenNr}%
\newcommand{\lstsource}[6]%
{%
  \setcounter{StartZeilenNr}{#2}%
  \addtocounter{StartZeilenNr}{0}%
  \lstset%
  {%
    language={#4},%
    numbers={left},%
    escapechar={§},%
    basicstyle={\ttfamily\scriptsize},%
    firstnumber={#2},%
    breaklines=true,%
    breakatwhitespace=true,%
    frame={tb},%
%    framerule={0pt},%
    framextopmargin={1mm},%
    xleftmargin={3mm},%
    framexleftmargin={3mm},%
    belowcaptionskip={5pt},%
    abovecaptionskip={4mm},%
    keywordstyle={\color{blue}},%
    commentstyle={\color{ForestGreen}},%
    stringstyle={\color{BrickRed}},%
    literate=%
    {Ö}{{\"O}}1%
    {Ä}{{\"A}}1%
    {Ü}{{\"U}}1%
    {ß}{{\ss}}1%
    {ü}{{\"u}}1%
    {ä}{{\"a}}1%
    {ö}{{\"o}}1%
  }%
\lstdefinestyle{numbers}%
  {numbers=left,numbersep=3mm,%
   numberstyle=\color[gray]{0.5},%
   xleftmargin=5mm,framexleftmargin=5mm}
\lstdefinestyle{caption}%
  {frame=b,framerule=0pt,%
   framexbottommargin=1mm}
  \lstinputlisting[style=numbers,caption={#5},label={#6},firstline={#2},lastline={#3}]{#1}%
}%

%Aufgaben
\newtheorem{aufgabe}{Aufgabe}[chapter]


\newcommand{\myfigure}[4]{%
\renewcommand{\figurename}{Abb.}
 \includegraphics[scale=#2,clip=true]{#1}%
 \caption{#3}\label{#4}%
\renewcommand{\figurename}{Abbildung}
}

%Bilder
\newenvironment{ebild}[4]
{
	\begin{figure}[hbtp]
		\begin{center}%
			\includegraphics[scale=#2,clip=true]{#1}%
			\caption{#3}\label{#4}%
		\end{center}%
	}
	{
	\end{figure}
}
% #1 = file
% #2 = scale
% #3 = caption
% #4 = label
\newcommand{\myebild}[4]{\begin{ebild}{#1}{#2}{#3}{#4} \end{ebild}}


\newenvironment{ezweihbild}[8]
        {%
                \begin{figure}[hbtp]%
                        \centering%
                        \begin{minipage}[b]{6.5cm}%
                                \centering%
                                \includegraphics[scale=#2]{#1}%
                                \caption{#3}\label{#4}%
                        \end{minipage}%
                \hfil%
                        \begin{minipage}[b]{6.5cm}%
                                \centering%
                                \includegraphics[scale=#6]{#5}%
                                \caption{#7}\label{#8}%
                        \end{minipage}%
        }
        {
                \end{figure}%
        }
\newcommand{\myezweihbild}[8]{\begin{ezweihbild}{#1}{#2}{#3}{#4}{#5}{#6}{#7}{#8} \end{ezweihbild}}

\newenvironment{ezweivbild}[8]
        {%
                \begin{figure}[hbtp]%
                        \centering%
                        \begin{minipage}[b]{13cm}%
                                \centering%
                                \includegraphics[scale=#2]{#1}%
                                \caption{#3}\label{#4}%
                        \end{minipage}%
                \vfil%
                        \begin{minipage}[b]{13cm}%
                                \centering%
                                \includegraphics[scale=#6]{#5}%
                                \caption{#7}\label{#8}%
                        \end{minipage}%
        }
        {
                \end{figure}%
        }
\newcommand{\myezweivbild}[8]{\begin{ezweivbild}{#1}{#2}{#3}{#4}{#5}{#6}{#7}{#8} \end{ezweivbild}}

\DeclareDocumentCommand{\newdualentry}{ O{} O{} m m m m } {
  \newglossaryentry{gls-#3}{name={#5},text={#5\glsadd{#3}},
    description={#6},#1
  }
  \makeglossaries
  \newacronym[see={[Glossar:]{gls-#3}},#2]{#3}{#4}{#5\glsadd{gls-#3}}
}

