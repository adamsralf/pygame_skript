\newglossaryentry{pygame}
{
  name={Pygame},
  description={Pygame ist ein Verbund von Modulen, der die Entwicklung von Computerspielen in Python unterstützt}
}

\newglossaryentry{python}
{
  name={Python},
  description={Python ist eine höhere Interpretersprache mit prozeduralen und objektorientierten Paradigmen. Sie wurde 1991 von Guido van Rossum entwickelt und erfreut sich derzeit größter Beliebtheit}
}

\newglossaryentry{konstante}
{
  name={Konstante},
  plural={Konstanten},
  description={Eine Konstante ist ein Wert, der zur Laufzeit eines Programmes nicht mehr geändert werden kann. In vielen Programmiersprachen können Variablen durch Schlüsselwörter wie \texttt{const} als Konstanten -- also Unveränderlichen -- deklariert werden. Direkte beispielsweise Zahlen- oder Stringangaben im Quelltext sind ebenfalls Konstanten}
}

\newglossaryentry{funktion}
{
  name={Funktion},
  plural={Funktionen},
  description={Eine Funktion ist in der Programmierung ein Anweisungsblock mit einem Namen. Sie können Parametersätze haben und Ergebnisse zurückliefern. In der Regel gilt dabei das Prinzip, dass alle Werte innerhalb der Funktion lokal sind}
}

\newglossaryentry{klasse}
{
  name={Klasse},
  plural={Klassen},
  description={Eine Klasse beschreibt die Attribute und die Methoden (Funktionen) einer inhaltlich abgeschlossenen Programmiereinheit. In der Praxis gibt es viele Varianten von Klassen, aber im Prinzip wird definiert, welche Informationen eine Klasse ausmacht (z.B. Marke, Farbe und Baujahr eines Autos) und was man mit einem Objekt der Klasse alles tun kann (z.B. beschleunigen, kaufen und tanken beim einem Auto). Die Informationen werden \emph{Attribute} genannt und die Möglichkeiten \emph{Methoden} oder \emph{member funtions}} 
}

\newglossaryentry{namensraum}
{
  name={Namensraum},
  plural={Namensräume},
  description={Innerhalb eines Namensraums müssen alle Namen für Klassen, Funktionen und Konstanten eindeutig sein. In der Regel werden Namensräume in Python anhand der Module und Pakete definiert} 
}

\newglossaryentry{umgebungsvari}
{
  name={Umgebungsvariable},
  plural={Umgebungsvariablen},
  description={Dies sind Variablen, die nicht vom Programm, sondern von der Programmumgebung verwaltet werden. Die Programmumgebung kann das Betriebssystem sein, aber auch eine Server. Über Umgebungsvariablen kann die Umgebung mit meinem Programm Informationen austauschen. In unserem Beispiel wird der Fensterverwaltung bzw. dem Betriebssystem mitgeteilt, an welcher Koordinate die linke obere Ecke des Fensters auf dem Bildschirm erscheinen soll} 
}


\newglossaryentry{bitmap}
{
  name={Bitmap},
  plural={Bitmaps},
  description={Der Begriff Bitmap hat hier zwei Bedeutungsebenen: Allgemein meint er Farb- und Transparenzinformationen eines Bildes in einer Datei. Typische Beispiele sind Dateien im Format \Gls{jpg}, \Gls{png} oder \Gls{bmp}. Im Speziellen ist damit das Bitmap-Dateiformat zur Bildspeicherung (Windows Bitmap, BMP) gemeint} 
}

\newglossaryentry{mainloop}
{
  name={Hauptprogrammschleife},
  plural={Hauptprogrammschleifen},
  description={Jedes nichttriviale Programm muss entscheiden, ob es noch weiterlaufen soll, oder ob die Verarbeitung beendet werden kann. Falls die Verarbeitung noch nicht beendet werden kann oder soll, muss mit der Benutzerinteraktion oder anderen Programmfunktionen fortgefahren werden und zwar solange, bis das Programm beendet werden kann oder soll. Dies wird in der Regel durch eine Hauptprogrammschleife gesteuert. Beispiele: Das Betriebssystem läuft, solange bis es heruntergefahren wird. Die Windows-Anwendung läuft, bis ALT+F4 betätigt wurde} 
}

\newglossaryentry{doublebuffer}
{
  name={Doublebuffer},
  description={Dies ist ein zweiter Speicherbereich, der genauso groß ist wie der Bildschirmspeicher. Wird jetzt etwas auf die Spielfläche gezeichnet, passiert dies zunächst auf diesem zweiten Speicher. Erst wenn alle Spielelemente ihr neues Aussehen gemalt haben, wird mit einem Schlag der alte Bildschirmspeicher mit dem zweiten ausgetauscht. Bei bestimmten Hardware- oder Grafikkonfigurationen kann es passieren, dass der Bildschirmspeicher neu gemalt wird, obwohl das Spiel noch nicht alle neuen Zustände abgebildet hat. Dadurch können hässliche Artefakte entstehen. Durch das Doublebuffering wird dieser Effekt vermieden} 
}

\newglossaryentry{messagequeue}
{
  name={Message Queue},
  description={Warteschlange des Betriebssystem zur Verwaltung von Ereignissen, die vom System erzeugt oder empfangen wurden. Laufende Anwendungen können diese Nachrichten für sich deklarieren und aus der Warteschlange entnehmen} 
}

\newglossaryentry{flag}
{
  name={Flag},
  description={Eine meist boolsche Variable, die eine Operation/Schleife ein- und ausschaltet} 
}

\newglossaryentry{alpha}
{
  name={Alpha-Kanal},
  description={Für jedes Pixel eines Bildes werden Farbinformationen meist im \glslink{rgb}{RGB}-Format abgespeichert: R-Kanal, G-Kanal und B-Kanal. Durch eine zusätzliche Information kann man noch angeben, wie durchscheinend das Pixel sein soll. Diese zusätzlich Informationen nennt man den Alpha-Kanal} 
}

\newglossaryentry{semantik}
{
  name={Semantik},
  description={Bedeutung einer Angabe. Wird meist in Abgrenzung zu \Gls{syntax} einer Angabe verwendet} 
}

\newglossaryentry{syntax}
{
  name={Syntax},
  description={Form oder Grammatik einer Angabe. Wird meist in Abgrenzung zur \Gls{semantik} einer Angabe verwendet} 
}

\newglossaryentry{polygon}
{
  name={Polygon},
  description={Ein geschlossener \gls{linienzug}. Wird meist durch eine Folge von Punkten definiert, wobei der letzte Punkt mit dem ersten verbunden wird} 
}

\newglossaryentry{linienzug}
{
  name={Linienzug},
  description={Eine Folge miteinander verbundener Linien. Wird meist durch eine Folge von Punkten definiert. Bei einem geschlossenen Linienzug spricht man von einem \gls{polygon}} 
}

\newglossaryentry{aequidistant}
{
	name={\"aquidistant},
	description={Der Abstand von Elementen ist immer der gleiche. Bei gleich großen Elementen bedeutet dies, dass der Platz zwischen diesen immer gleich ist. Bei nicht gleichgroßen Elementen muss es einen Bezugspunkt geben. Sollen die Mittelpunkte der Elemente immer die gleiche Distanz haben, oder sollen der rechte Rand des einen immer den gleichen Abstand zum linken Rand des nächsten haben? Auch wird zwischen horizontaler und vertikaler Äquidistanz unterschieden} 
}

\newglossaryentry{kollsionserkennung}
 {
 	name={Kollsionserkennung},
 	description={Überprüfung, ob zwei Bitmaps sich in irgendeiner einer Art und Weise \emph{berühren}. In Pygame nutzen wir drei Arten der Kollisionserkennung: Schneiden sich die umgebenden Rechtecke der Bitmaps, schneiden sich die Innenkreise der Bitmaps und haben nicht-transparente Pixel der Bitmaps die selbe Koordinate} 
 }
 
\newglossaryentry{framework}
 {
 	name={Framework},
 	description={In der Informatik ist damit eine Arbeitsumgebung gemeint. Dies können einzelne Klassen, Funktionsbibliotheken oder ganze \Gls{ide} sein} 
 }
 
\newglossaryentry{kindklasse}
 {
 	name={Kindklasse},
 	description={Die Spezialisierung einer Elternklasse. Sie erbt nicht private Attribute und Methoden und kann diese dann so verwenden, als wenn es die eigenen wären} 
 }

\newglossaryentry{boss}
{
	name={Boss-Taste},
	description={Bei Betätigung der Boss-Taste wird das Spiel ohne Rückfragen so schnell wie möglich beendet. Der Boss kommt herein, der Lehrer steht hinter einem, ..} 
}

\newglossaryentry{render}
{
	name={Rendern},
	description={Das Erzeugen eines Bildes -- meist in Bitmap-Format -- aus einer Bildbeschreibungsangabe} 
}

\newglossaryentry{font}
{
	name={Font},
	description={In digitaler Form vorhandene Information über einen Zeichensatz. Er ist meist in einem dieser drei Formate verfügbar: als Bitmap, als Vektorgrafik oder als Beschreibung} 
}

\newglossaryentry{dictionary}
{
	name={Dictionary},
	description={Eine Datenstruktur, welche Werte unter einem einzigartigen Schlüssel ablegt. Andere Namen sind: \index{Zuordnungstabelle}Zuordnungstabelle, \index{assoziatives Array}assoziatives Array, \index{Hashtable}Hashtable} 
}

\newglossaryentry{array}
{
	name={Array},
    plural={Arrays},
	description={Eine Datenstruktur, welche Werte unter einem einzigartigen Index (meist eine positive ganze Zahl) ablegt. Im engeren Sinne enthalten Array immer nur Elemente des gleichen Datentyps. Bei Sprachen wie PHP oder Python gilt das nicht} 
}

\newglossaryentry{slicing}
{
	name={Slicing},
	description={Eine Technik, mit deren Hilfe man Teilmengen aus Strings oder Arrays bequem ausschneiden oder extrahieren kann} 
}

\newglossaryentry{unicode}
{
	name={Unicode},
	description={Ein Verfahren zur Kodierung von Zeichen und Symbolen. Gängige Umsetzungen sind UTF-8, UTF-16 und UTF-32} 
}

\newglossaryentry{sprite}
{
	name={Sprite},
    plural={Sprites},
	description={Ein Grafikobjekt, welches auf einem Hintergrund platziert wird und meist auch Eigenschaften hat, die über die reine Anzeige hinausgehen. So können Sprites sich oft bewegen oder werden animiert oder lösen bei Kontakt eine Reaktion aus. Üblicherweise meint man damit immer 2D-Objekte. Andere Namen sind \emph{moveable object (MOB)} oder \emph{blitter object (BOB)} } 
}

\newglossaryentry{spritelib}
{
	name={Spritelib},
	description={Meist eine Grafikdatei im Bitmap-Format, welches viele einzelne \glspl{sprite} enthält} 
}

\newglossaryentry{maske}
{
	name={Maske},
	description={Ein Maske (engl. mask) ist ein Bitmap, welches die wichtigen von den unwichtigen Pixel eines Sprites unterscheidbar macht. Bei Sprites mit Transparenzen kann die Maske einfach dadurch ermittelt werden, dass alle transparenten Pixel unwichtig sind. Um Speicherplatz und Rechenzeit zu sparen, werden die Masken oft nicht in den üblichen Bitmap-Formaten abgelegt, sondern Bit für Bit. Ein Byte kann also die Maskeninformation für 8~Pixel kodieren} 
}

\newglossaryentry{listcomp}
{
	name={List Comprehension},
	description={In Python kann man den Inhalte einer Liste, eines Tupels, eines Arrays oder eines Dictionarys nicht nur durch explizite Vorgaben festelegen, sondern auch indem man eine Generierungsvorschrift formuliert: \texttt{squares = [x**2 for x in range(10)]}} 
}

\newglossaryentry{endlosschleife}
{
	name={Endlosschleife},
	description={In der Informatik ist eine Endlosschleife eine Folge von Anweisungen, die sich immer wiederholt und für die es kein definiertes Abbruchkriterium gibt. In den meisten Fällen sind Endlosschleifen nicht gewollt und damit ein Fehler in der Anwendung. Sie entstehen oft durch fehlerhafte Schleifenbedingungen. Endlosschleifen werden manchmal aber auch gezielt eingesetzt: \texttt{while True:}} 
}

\newglossaryentry{toggling}
{
	name={Toggling},
	description={In der Informatik bedeutet dies, dass der Wert einer Boolschen Variable von \true\ nach \false\ bzw. von \false\ nach \true\ wechselt: \emph{to toggle} = \emph{umschalten}} 
}

\newglossaryentry{pythagoras}
{
	name={Satz des Pythagoras},
	description={In einem rechtwinkligen Dreieck ist die Summe der Kathetenquadrate gleich dem Hypotenusenquadrat: $c^2 = \sqrt{a^2 + b^2}$. Der Satz ist nach dem Mathematiker \emph{Pythagoras von Samos} (um 570 v.Chr. bis um 510 v.Chr.) benannt} 
}

\newglossaryentry{pylance}
{
	name={Pylance},
	description={Pylance ist die Standard Python-Erweiterung von Visual Code zur Unterstützung der Python-Programmierung. Seine wensentlichen Features sind die Typüberwachung und Autovervollständigung} 
}

\newglossaryentry{grad}
{
	name={Grad (°)},
	description={Maßeinheit für einen Winkel. Der Vollkreis hat dabei $360°$} 
}

\newglossaryentry{radiant}
{
	name={Radiant (rad)},
	description={Maßeinheit für einen Winkel. Der Vollkreis hat dabei $2\pi rad$} 
}

\newglossaryentry{ogg}
{
	name={ogg},
	description={Kodierung von Sound-Dateien. Kommt vom englischen \emph{to ogg}. Ziel war eine lizenfreie, einfache und effiziente Kodierung von Sound} 
}

\newglossaryentry{mp3}
{
	name={mp3},
	description={Abkürzung von \emph{ISO MPEG Audio Layer 3}. Eine im wesentlichen vom deutschen Elektrotechnikingenieur und Mathematiker Karlheinz Brandenburg entwickeltes Kodierungs- und Kompressionsverfahren von Sound und Musik} 
}

\newglossaryentry{fade}
{
	name={fade},
	description={Kommt vom englischen \emph{to fade} für \emph{verblassen}. In der Musik und bei Grafiken unterscheidet man einen \emph{fadein} und einen \emph{fadeout}. Bei einem fadein erscheint das Bild langsam bzw. wird die Lautstärke bei 0 beginnend auf die Ziellautstärke erhöht. Ein fadeout tut das Gegenteil} 
}

\newglossaryentry{singleton}
{
	name={Singleton},
	description={Ein Design-Pattern, welches sicherstellt, dass es immer nur ein Objekt einer Klasse gibt. Dieses wird dann meist (halb-)öffentlich zur Verfügung gestellt. Das Singleton ist wegen seiner konzeptionellen Nähe zu globalen Variablen umstritten} 
}

\newglossaryentry{tradeoff}
{
    name={Trade-off},
    description={Jeder Vorteil wird durch einen Nachteil erkauft. Algorithmisch muss dann anhand der Datenlage abgewägt werden, ob in der Gesamtbetrachtung der Nutzen die Kosten überwiegt. Beispiel: Durch die Verwendung von Indizes werden Zugriffe auf Datenbankinhalte dramatisch beschleunigt (Nutzen). Um diese Beschleunigung zu erreichen, müssen Dateien angelegt werden, und das Anlegen, Ändern und Löschen von Daten wird langsamer, da diese Dateien dann mitgepflegt werden müssen (Kosten)} 
}

\newglossaryentry{float}
{
    name={Fließkommazahl},
    description={Bei einer Fließkommazahl werden Zahlen als Summen von Zweierpotenzen dargestellt, wobei der Exponent auch negativ sein kein. Beispiel: $6,75 = 2^2 + 2^1 + 2^1{-1} + 2^{-2}$. Da der Speicherplatz beschränkt ist oder es für bestimmte Zahlen keine endliche Darstellung gibt, bricht die Summenbildung zu irgendeinem Zeitpunkt ab. Die dabei nicht mehr berücksichtigten Summanden führen zu den Rundungsfehlern}
}

\newglossaryentry{int}
{
    name={Ganzzahl},
    description={Bei einer Ganzkommazahl werden Zahlen als Summen von Zweierpotenzen dargestellt, wobei der Exponent Null oder positiv ist. Beispiel: $17 = 2^4 + 2^0$. Der Wertebereich ist durch den Speicherplatz, den man einer ganzen Zahl zur Verfügung stellt, definiert. Stehen~$n$ Bits zur Verfügung, können ohne Vorzeichen Zahlen im Bereich $[0, 2^n-1]$ und mit Vorzeichen im Bereich $[2^{n-1}, 2^{n-1}-1]$ dargestellt werden} 
}


\newglossaryentry{signatur}
{
	name={Signatur},
	description={Die Signatur einer Funktion/Methode beschreibt die formalen Eigenschaften, die von einer Funktion/Methode von außen sichtbar ist. Dazu gehören die Sichtbarkeit, der Rückgabetyp, der Name und die Übergabeparameter} 
}



%\newacronym{mAn}{m.A.n.}{Meiner Ansicht nach}

\newdualentry{Stereo} % label
  {Stereo}            % abbreviation
  {Stereofonie}  % long form
  {Verfahren um mit mehr als einer Schallquelle einen mehrdimensionalen Schalleindruck zu erzeugen} % description

\newdualentry{PX} % label
  {px}            % abbreviation
  {Pixel}  % long form
  {Die kleinste bei gegebener Auflösung ansteuerbare Bildschirmfläche} % description

\newdualentry{bmp} % label
  {bmp}            % abbreviation
  {Windows Bitmap Format}  % long form
  {Bildinformationen im Windows Bitmap-Format} % description

\newdualentry{jpg} % label
  {jpeg}            % abbreviation
  {Joint Photographic Experts Group}  % long form
  {Verlustbehaftete komprimierte Bildinformationen} % description

\newdualentry{png} % label
  {png}            % abbreviation
  {Portable Network Graphics}  % long form
  {Verlustfrei komprimierte Bildinformationen} % description

\newdualentry{usb} % label
  {USB}            % abbreviation
  {Universal Serial Bus}  % long form
  {Bitserielles Datenübertragungsprotokoll} % description

\newdualentry{ssd} % label
  {SSD}            % abbreviation
  {Solid-State-Drive}  % long form
  {Festspeicherplattentechnologie, welche nicht auf magnetische Prinzipien, sondern auf Halbleitertechnik basiert} % description

\newdualentry{rgb} % label
  {RGB}            % abbreviation
  {Red Green Blue}  % long form
  {Additive Farbkodierung} % description

\newdualentry{fps} % label
  {fps}            % abbreviation
  {frames per second}  % long form
  {Maximale Anzahl der Bilder pro Sekunde} % description

\newdualentry{ide} % label
  {IDE}            % abbreviation
  {Integrated Development Environment}  % long form
  {Integrierte Entwicklungsumgebung. Diese heißen \emph{integriert}, da sie nicht nur einen Compiler und Linker enthalten, sondern auch einen Editor, Debugger, Profiler etc} % description

\newdualentry{srp} % label
  {SRP}            % abbreviation
  {Single Responsibility Principle}  % long form
  {Jede Klasse / jede Funktion sollte nur eine Verantwortlichkeit haben. Die Klasse / die Funktion sollte sich
  auf diese Aufgabe konzentrieren. Kapseln Sie eine Lösung in eine Klasse oder eine Methode} % description

\newdualentry{oo} % label
  {OO}            % abbreviation
  {Objektorientiert}  % long form
  {Die Analayse, das Design oder die Implementierung entspricht den allgemeinen Vorgaben der Objektorientierung} % description

\newdualentry{pt} % label
  {pt}            % abbreviation
  {DTP-Punkt}  % long form
  {Maßeinheit für Schriftgrößen} % description

\newdualentry{ttf} % label
  {ttf}            % abbreviation
  {True Type Font}  % long form
  {Die Schriftinformation wird nicht im Bitmap-Format, sondern in einer Art Vektorgrafikformat abgespeichert. Dadurch lassen sich \emph{beliebige} Schriftgrößen generieren} % description

\newdualentry{ms} % label
  {ms}            % abbreviation
  {Millisekunden}  % long form
  {Der $1/1000$ Teil eines Sekunde} % description

