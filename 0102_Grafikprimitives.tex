\newpage
%%%%%%%%%%%%%%%%%%%%%%%%%%%%%%%%%%%%%%%%%%%%%%%%%%%%%%%%%%%%%%%%%%%%%%%%%%%
\section{Grafikprimitive}\index{Grafikprimitive}
\subsection{Grundlagen}
Unter Grafikprimitiven versteht man gezeichnete einfache grafische Figuren wie Linien, Punkte, Kreise etc. Sie spielen in der Spieleprogrammierung nicht so eine große Rolle, können aber manchmal ganz nützlich sein. Ich will hier deshalb nur einige vorstellen. 

\lstsource{SRC/00 Einführung/02 Primitive/primitives00.py}{1}{44}{python}{Mein zweites \emph{Spiel}, Version 1.0}{srcPrimitives00}

Der Grundaufbau ist der gleiche wie in \srcref[vref]{srcStart01}. Die Unterschiede beginnen in \zeiref{srcPrimitives01}. Die Klasse \texttt{pygame.Color}\randnotiz{Color}\myindex{pyg}{\texttt{Color}|underline} kann Farbinformationen in verschiedenen Formaten inklusive eines \glslink{alpha}{Alpha-Kanals}\index{Alpha-Kanal} (Transparenz) kodieren; mehr dazu später in \abschnittref[vref]{secBitmapLaden}. Ich verwende hier eine RGB-Kodierung mit Farbkanalwerten zwischen~0 und~255. 

Für die meisten Fälle brauche ich mir aber keine eigenen Farben zu definieren. Pygame stellt mir eine wirklich umfangreiche Liste von~664 vordefinierten Farbnamen zur Verfügung\randnotiz{Farbnamen}\index{Farbnamen}. Überall dort, wo Farbwerte erwartet werden, kann ich entweder eine \texttt{Color}-Objekt, einen Zahlencode oder einen Farbnamen als String übergeben.

\begin{wrapfigure}[18]{r}{7.8cm}%
	\begin{center}%
		\vspace{-1cm}%
		\myfigure{primitives.png}{0.4}{Einige Grafikprimitve}{picPrimitive}%
	\end{center}%
\end{wrapfigure}%
Gehen wir der Reihe nach die einzelnen Figuren durch und fangen mit dem Rechteck an. Es gibt mehrere Möglichkeiten, ein Rechteck in Pygame zu bestimmen. Da wir es später auch sehr oft brauchen, möchte ich hier schonmal die Klasse \texttt{pygame.rect.Rect}\myindex{pyg}{\texttt{Rect}|underline}\randnotiz{Rect} einführen. Sie wird durch vier Parameter bestimmt: die linke obere Ecke, seine Breite und seine Höhe. In \zeiref{srcPrimitives02} wird also ein Rechteck an der Position $(10,10)$ mit der Breite von $20~px$ und einer Höhe von $30~px$ definiert. 

Hinweis: Die Klasse \texttt{Rect} ist kein gezeichnetes Rechteck, sondern lediglich ein Container für Informationen, die für ein Rechteck interessant sind. 

In \zeiref{srcPrimitives03} zeichnet \texttt{pygame.draw.rect()}\myindex{pyg}{\texttt{draw}!\texttt{rect()}}\randnotiz{rect()} ein gefülltes Rechteck. Die \Gls{semantik} der Parameter sollte selbsterklärend sein. Anders der Aufruf von \zeiref{srcPrimitives04}. Der erste Parameter hinter dem Rechteck -- hier \texttt{3} -- legt die Dicke der Linie fest. Ist dieser Parameter angegeben und größer~0, so wird das Rechteck nicht mehr ausgefüllt. Der Wert~\texttt{10} legt die Rundung der Ecken fest. Dort kann ein Wert von 0 bis $min(width, height)/2$ stehen, enspricht er doch dem Radius der Eckenrundung.

Allgemeiner als ein Rechteck ist ein \Gls{polygon}. Ein Polygon ist ein geschlossener Lienenzug, der in Pygame durch seine Punkte (Ecken) definiert wird. Ähnlich wie bei den Rechtecken, gibt es gefüllte (\zeiref{srcPrimitives06}) und ungefüllte (\zeiref{srcPrimitives07}) Varianten. Beide werden mit Hilfe von \texttt{pygame.draw.polygon()}\randnotiz{polygon()}\myindex{pyg}{\texttt{draw}!\texttt{polygon()}} gezeichnet. Vorsicht bei der Liniendicke: Diese wachsen nach außen, so dass bald hässliche Versatzstücke an den Ecken erkennbar werden. Probieren Sie es aus, indem Sie den Wert~\texttt{2} in~\texttt{5} ändern.

Für einzelne Linien gibt es \texttt{pygame.draw.line()}\myindex{pyg}{\texttt{draw}!\texttt{line()}}\randnotiz{line()} bzw. für einen -- hier ohne Beispiel -- \gls{linienzug} \texttt{pygame.draw.lines()}\myindex{pyg}{\texttt{draw}!\texttt{lines()}}\randnotiz{lines()}. Ein Beispiel finden Sie in \zeiref{srcPrimitives08}.

Ein Kreis wird durch zwei Angaben definiert: Mittelpunkt und Radius. In \zeiref{srcPrimitives09} wird mit \texttt{pygame.draw.circle()}\myindex{pyg}{\texttt{draw}!\texttt{circle()}}\randnotiz{circle()} ein gefüllter Kreis mit dem Mittelpunkt $(40, 150)$ und einem Radius von $30~px$ gezeichnet. Wie bei Rechtecken und Polygonen gibt es auch nicht gefüllte Varianten (\zeiref{srcPrimitives10}). Interessant ist der Kreisbogenausschnitt in \zeiref{srcPrimitives11}. Hier wird über boolsche Variablen gesteuert, welcher Abschnitt des Kreisbogens gezeichnet wird (Näheres in der Pygame-Referenz).

Zum Schluss noch einen kleine Farbenspielerei. Seltsamerweise gibt es in Pygame keine eigene Funktion zum Zeichnen eines einzelnen Punktes/Pixel. Ich habe hier mal drei Workarounds programmiert, die ich gefunden habe. Man könnte sich noch weitere überlegen: Eine Linie mit $start=ende$, ein Kreis mit dem Radius~1 usw.

In \zeiref{srcPrimitives12} wird der Punkt durch das Setzen eines einzelnen Farbwertes an einer Position mit \texttt{pygame.Surface.set\_at()}\myindex{pyg}{\texttt{Surface}!\texttt{set\_at()}}\randnotiz{set\_at()} gezeichnet. Man könnte auch die schon oben verwendete Surface-Funktion \texttt{fill()} mit einer Ausdehnung von nur einem Pixel Breite und Höhe verwenden (\zeiref{srcPrimitives13}). Ein Möglichkeit einen Pixel über eine Grafikbibliothek zu setzen, ist die experimentelle \texttt{gfxdraw}-Umgebung. In \zeiref{srcPrimitives14} wird mit \texttt{pygame.gfxdraw.pixel()}\myindex{pyg}{\texttt{gfxdraw}!\texttt{pixel()}}\randnotiz{pixel()} ein einzelnes Pixel gesetzt. Die \texttt{gfxdraw}-Umgebung wird nicht automatisch durch \texttt{import pygame} importiert (siehe \zeiref{srcPrimitives15}).

\subsection{Beispiel: Kreise}

Man kann mit Grafikprimitiven dynamische Effekte einbauen, wie beispielsweise Partikelschwärme. Ich will hier mal ein super einfaches Beispiel für einen mausgesteuerten Schwarm aus Kreisen vorstellen.

\begin{wrapfigure}[20]{r}{6.0cm}%
	\begin{center}%
		\vspace{-1cm}%
		\myfigure{circles01.png}{0.6}{Sicher keine\\Partikelfontaine}{picCircles01}%
	\end{center}%
\end{wrapfigure}%
Bauen wir zuerst ein kleines Programm, dass an der Mausposition einen Kreis zeichnet. Die Klasse \texttt{Circle} (siehe \zeiref{srcCircles0101}) enthält alle Informationen, die ich für das Zeichnen von Kreisen brauche: Position, Radius und Farbe. Die Position wird per Übergabeparameter bestimmt. In der Methode  \texttt{draw()} wird die Bildschirmausgabe gekapselt.

\texttt{main()} enthält nun sehr viel bekanntes, aber auch ein paar Neuigkeiten. In \zeiref{srcCircles0101} wird die Bildschirmgröße in einer Liste vorgehalten, da wir die Info noch an anderer Stelle als in \zeiref{srcCircles0103} brauchen. Darunter wird in \zeiref{srcCircles0104} eine Liste für die Aufnahme der Kreise definiert. 

In der Hauptprogrammschleife wird in \zeiref{srcCircles0105} abgefragt, ob die linke Maustaste\randnotiz{\texttt{get\_pressed(})}\myindex{pyg}{\texttt{mouse}!\texttt{get\_pressed()}|underline} gedrückt wurde. Wenn ja, wird ein Kreis an der Mausposition\randnotiz{\texttt{get\_pos(})}\myindex{pyg}{\texttt{mouse}!\texttt{get\_pos()}} erzeugt. Anschließend wird der Bildschirm mit weißer Farbe aufgefüllt und die Kreise des Kontainers werden gemalt.

Das Ergebnis ist noch wenig berauschend (siehe \abbref[vref]{picCircles01}) und erinnert mehr an ein Malprogramm wie Paint.

\lstsource{SRC/00 Einführung/02 Primitive/circles01.py}{1}{999}{python}{Partikelfontaine, Version 1.0}{srcCircles01}

\begin{wrapfigure}[10]{r}{6.0cm}%
	\begin{center}%
		\vspace{-1cm}%
		\myfigure{circles02a.png}{0.6}{Version 2}{picCircles02}%
	\end{center}%
\end{wrapfigure}%
Im nächsten Schritt wollen wir aus den klobigen Kreisen bunte Partikel machen. Auch sollen diese nicht genau auf der Mausposition landen, sondern darum verstreut. Dazu müssen nur minimal Änderungen in der Klasse \texttt{Circle} vorgenommen werden.

Die beiden Positionsangaben werden nun durch eine Zufallszahl zwischen~$-2$ und~$+2$ ergänzt. Auch wird der Radius auf~$2~px$ reduziert. Die Farbe wird ebenfalls durch zufällige Werte gestreut. Hier habe ich einige Kombinationen ausprobiert und mit gefällt diese Farbvariation ganz gut. Spielen Sie ruhig selber mal mit den Farbkanälen und den Zufallswerten rum. Das Ergebnis in \abbref[vref]{picCircles02} sieht schon besser aus. 

\lstsource{SRC/00 Einführung/02 Primitive/circles02.py}{7}{12}{python}{Partikelfontaine, Version 2.0}{srcCircles02}

Nun wollen wir ein wenig Dynamik ins Spiel bringen. Die Partikel sollen zuerst nach oben steigen und dann nach unten fallen. Dazu habe ich in \texttt{Circle} die vertikale Geschwindigkeit \texttt{speedy} hinzugefügt und mit einem zufälligen Startwert versehen (\zeiref{srcCircles0301}). Die Division durch~$10.1$ sorgt dafür, dass keine glatten Werte entstehen. Spielen Sie auch hier mal mit den Werten rum, um die Effekte zu sehen.

Auch muss die Klasse um die Methode \texttt{update()} erweitert werden. In dieser Methode wird die neue vertikale Position \texttt{posy} anhand der vertikalen Geschwindigkeit \texttt{speedy} berechnet und die Geschwindigkeit wiederum bzgl. der Schwerkraftn \randnotiz{Schwerkraft}\index{Schwerkraft}\index{gravity}\texttt{gravity} verändert. Damit alle Partikel immer der gleichen Schwerkraft unterliegen, habe ich \texttt{gravity} als statisches Attribut definiert (\zeiref{srcCircles0302}).

\lstsource{SRC/00 Einführung/02 Primitive/circles03.py}{7}{19}{python}{Partikelfontaine, Version 3.0, Klasse \texttt{Circle}}{srcCircles03a}

Verbleibt noch der Aufruf von \texttt{update()} in der Hauptprogrammschleife.

\lstsource{SRC/00 Einführung/02 Primitive/circles03.py}{45}{50}{python}{Partikelfontaine, Version 3.0, Aufruf von \texttt{update()}}{srcCircles03b}

So richtig spritzig ist die Fontaine aber immer noch nicht. Streuen wir also die Partikel auch horizontal. Im Konstruktor wird dazu das Attribut \texttt{speedx} hinzugefügt. Die obere und untere Grenze des Zufallszahlengenerators bestimmen die Breite der Partikelfontaine. Probieren Sie hier Werte aus, die ihrer Ästhetik entsprechen. In \texttt{update()} muss dann die neue horizontale Position \texttt{posx} berechnet werden. 

Die horizontale Geschwindigkeit muss nicht angepasst werden, da \texttt{gravity} nur nach unten wirken soll.

\lstsource{SRC/00 Einführung/02 Primitive/circles04.py}{10}{21}{python}{Partikelfontaine, Version 4.0, \texttt{Circle.update()}}{srcCircles04}

\begin{wrapfigure}[17]{r}{6.0cm}%
	\begin{center}%
		\vspace{-1cm}%
		\myfigure{circles05.png}{0.6}{Partikelfontaine, \\Version 5: fast fertig}{picCircles05}%
	\end{center}%
\end{wrapfigure}%
Die Liste \texttt{circles} enthält nach einiger Zeit viele Partikel, die überhaupt nicht mehr angezeigt werden. Wir wollen diese löschen. Dazu soll in der Klasse \texttt{Circle} festgestellt werden, ob man gelöscht werden kann. Im ersten Schritt fügen wir der Klasse das Löschflag \texttt{todelete} hinzu (siehe \zeiref{srcCircles0500}), welches auf \false\ initialisiert wird; ein neuer Partikel soll natürlich nicht sofort gelöscht werden.

In \texttt{update()} wird dann nach der Berechnung der neuen Position überprüft, ob der Partikel zu löschen ist. Unser Kriterium soll sein, dass der Partikel den Bildschirm verlassen hat.

In \zeiref{srcCircles0501} wird überprüft, ob der rechte Rand des Partikels (Mittelpunkt plus Radius), links außerhalb des Bildschirm liegt. Falls ja, muss das Löschflag auf \true\ gesetzt werden. Analog werden in \zeiref{srcCircles0502} und \zeiref{srcCircles0503} der rechte und der untere Rand des Bildschirm überprüft. Dabei wird mit Hilfe der Methode \texttt{pygame.display.get\_window\_size()}\randnotiz{get\_window\_size()}\myindex{pyg}{\texttt{display}!\texttt{get\_window\_size()}|underline} jeweils die Breite bzw. Höhe des Bildschirms ermittelt. Diese Methode liefert mir an jedem Punkt meines Spiels die Bildschirmgröße als 2-Tupel wieder. Der nullte Wert ist dabei die Breite und der erste die Höhe. Ein Test, ob der Partikel nach oben verschwunden ist, wird nicht benötigt, da er ja irgendwann wieder runterfällt und damit wieder sichtbar wird.

\lstsource{SRC/00 Einführung/02 Primitive/circles05.py}{10}{28}{python}{Partikelfontaine, Version 5.0, Klasse \texttt{Circle}}{srcCircles05a}

Im Hauptprogramm muss ich nun einen passende Löschlogik implementieren. Vorab soll meine Fontaine aber noch mehr \emph{Wumms} bekommen: In \zeiref{srcCircles0504} wird nicht ein Partikel erzeugt, sondern immer gleich~5.

In \zeiref{srcCircles0505} wird eine leere Liste erstellt, die die zu löschenden Partikel enthalten wird. Innerhalb der Update-Schleife wird nun zusätzlich überprüft, ob der Partikel zu löschen ist (\zeiref{srcCircles0506}). Wenn ja, wird dieser Partikel in die Liste \texttt{todelete} aufgenommen. Nach Beendigung der Update-Schleife werden die zu löschenden Partikel ab \zeiref{srcCircles0507} aus der Liste \texttt{circles} entfernt.

In \abbref[vref]{picCircles05} können Sie eine Fontaine sehen. So richtig cool sieht das aber erst aus, wenn Sie die Maus dabei bewegen.

\lstsource{SRC/00 Einführung/02 Primitive/circles05.py}{45}{67}{python}{Partikelfontaine, Version 5.0, Hauptprogrammschleife}{srcCircles05b}

Warum rufe ich nicht den \texttt{remove()} schon innerhalb der Update-Schleife auf? Deshalb: \emph{Verlängern oder verkürzen Sie nie eine Liste, die Sie gerade durchwandern.} Es können höchst seltsame Effekte entstehen. Schätzen Sie mal die Anzahl der Schleifendurchgänge des folgendes Programmes.

\begin{lstlisting}[firstnumber=1]
klein = [1, 2, 3]
for a in klein:
    klein.append(a*10)
    print(klein)	
\end{lstlisting}

%\begin{wrapfigure}[17]{r}{6.0cm}%
%	\begin{center}%
%		\vspace{-1cm}%
%		\myfigure{circles06.png}{0.6}{Partikelfontaine, \\Version 6: fertig}{picCircles06}%
%	\end{center}%
%\end{wrapfigure}%
Kleine Änderungen der Parameter können schon interessante visuelle Effekte haben. Leider können die hier nicht so gut durch Abbildungen gezeigt werden, daher: Selbst programmieren und ausprobieren.


\lstsource{SRC/00 Einführung/02 Primitive/circles06.py}{1}{999}{python}{Partikelfontaine, Version 6.0}{srcCircles06}


\subsection*{Was war neu?}

Mit Hilfe von Grafikprimitiven können eigene Zeichnungen erstellt und verwendet werden. Sie stehen meist in einer gefüllten und ungefüllten Variante zur Verfügung. Farben können selbst definiert werden oder aus einer Liste von vordefinierten Farben ausgewählt werden. 

Es wurden folgende Pygame-Elemente eingeführt:

\begin{itemize}
	\item Vordefnierte Farbnamen:\index{Farbennamen}\\ 
	\url{http://www.pygame.org/docs/ref/color_list.html}

	\item \texttt{import pygame.gfxdraw}:\\ \url{https://www.pygame.org/docs/ref/gfxdraw.html}
	
	\item \texttt{pygame.Color}:
	\myindex{pyg}{\texttt{Color}}\\
	\url{https://www.pygame.org/docs/ref/color.html}
	
	\item \texttt{pygame.draw.circle()}:
	\myindex{pyg}{\texttt{draw}!\texttt{circle()}}\\
	\url{https://www.pygame.org/docs/ref/draw.html#pygame.draw.circle}
	
	\item \texttt{pygame.draw.line()}:
	\myindex{pyg}{\texttt{draw}!\texttt{line()}}\\
	\url{https://www.pygame.org/docs/ref/draw.html#pygame.draw.line}
	
	\item \texttt{pygame.draw.lines()}:
	\myindex{pyg}{\texttt{draw}!\texttt{lines()}}\\
	\url{https://www.pygame.org/docs/ref/draw.html#pygame.draw.lines}
	
	\item \texttt{pygame.draw.polygon()}:
	\myindex{pyg}{\texttt{draw}!\texttt{polygon()}}\\
	\url{https://www.pygame.org/docs/ref/draw.html#pygame.draw.polygon}
	
	\item \texttt{pygame.draw.rect()}:
	\myindex{pyg}{\texttt{draw}!\texttt{rect()}}\\
	\url{https://www.pygame.org/docs/ref/draw.html#pygame.draw.rect}
	
	\item \texttt{pygame.display.get\_window\_size()}:
	\myindex{pyg}{\texttt{display}!\texttt{get\_window\_size()}}\\
	\url{https://www.pygame.org/docs/ref/display.html#pygame.display.get_window_size}

	\item \texttt{pygame.gfxdraw.pixel()}:
    \myindex{pyg}{\texttt{gfxdraw}!\texttt{pixel()}}\\ \url{https://www.pygame.org/docs/ref/gfxdraw.html#pygame.gfxdraw.pixel}

	\item \texttt{pygame.mouse.get\_pos()}:
	\myindex{pyg}{\texttt{mouse}!\texttt{get\_pos()}}\\
	\url{https://www.pygame.org/docs/ref/mouse.html#pygame.mouse.get_pos}

	\item \texttt{pygame.mouse.get\_pressed()}:
	\myindex{pyg}{\texttt{mouse}!\texttt{get\_pressed()}}\\
	\url{https://www.pygame.org/docs/ref/mouse.html#pygame.mouse.get_pressed}

	\item \texttt{pygame.rect.Rect}:
    \myindex{pyg}{\texttt{rect}!\texttt{Rect}}\\
    \url{https://www.pygame.org/docs/ref/rect.html}
	
	\item \texttt{pygame.Surface.set\_at()}:
	\myindex{pyg}{\texttt{Surface}!\texttt{set\_at()}}\\
	\url{https://www.pygame.org/docs/ref/surface.html#pygame.Surface.set_at}
	
\end{itemize}

