\newpage
%%%%%%%%%%%%%%%%%%%%%%%%%%%%%%%%%%%%%%%%%%%%%%%%%%%%%%%%%%%%%%%%%%%%%%%%%%%
\section{Events}\label{secEvents}\index{Event}
Wir haben Ergeignisse (\gls{event}) schon an zwei Stellen verwendet, ohne sie näher betrachtet zu haben. Zum einen als wir in \kapref[vref]{secTastatur} über die Tastatur und zum anderen als wir in \kapref[vref]{secMaus} über die Maus gesprochen haben. 

Wir werden hier drei Aspekte näher beleuchten: 

\begin{itemize}
    \item Welche Infos stecken eigentlich in einem Event?
    \item Wie kann ich selbst ein Event erzeugen?
    \item Wie kann ich periodisch Ereignisse erzeugen lassen?
\end{itemize}


\subsection{Welche Infos stecken in einem Event?}

Das Programm zu \srcref[vref]{srcEvent00a} erstellt lediglich ein graues Fenster und gibt mit \texttt{print} in \zeiref{srcEvents0001} das Event in der Konsole aus.

\lstsource{SRC/00 Einführung/15 Events/events00.py}{38}{45}{python}{Events -- Informationen ausgeben}{srcEvent00a} 

Wandert man nun mit der Maus hin und her, drückt ein paar Tasten oder beendet die Anwendung, erscheint ungefähr sowas in der Konsole, wobei ich viele redundante Zeilen gelöscht habe:

\lstsource{console00.txt}{1}{999}{python}{Events -- Konsolenausgabe}{srcConsole00a} 

Zunächst fällt auf, dass die Eventinformationen in Form eines Dictionarys zur Verfügung gestellt werden. Den ersten Eintrag (die Nummer mit dem Bindestrich und anschließendem Namen) kann über \texttt{event.type}\index{Event!\texttt{type}} abgefragt werden. Damit man sich diese Nummern nicht auswendig merken muss, werden von Pygame entsprechende Konstanten angeboten; für die Tastatur finden Sie in \tabref[vref]{tabKey} und für die Maus in \tabref[vref]{tabMousebutton} eine Übersicht.

In den geschweiften Klammern stehen nun die Key/Value-Paare -- also die dem Event mitgegebenen Informationen. Bei Tastaturereignissen sind dies beispielsweise die Darstellung als Unicode-Zeichen oder seine Unicodenummer. Mausereignissen wird sinnvollerweise die Position und die Tastennummer mitgegeben. Das Klicken auf dem \emph{Fenster schließen}-Button oben rechts löst mehrere Ereignisse aus, hier die letzten vier der Liste.

Wir werden gleich sehen, dass bei selbsterstellten Ereignissen, diese Infos nach eigenem Bedarf definiert werden können.

\subsection{Wie kann ich selbst ein Event erzeugen?}

Als Beispiel will ich hier zwei primitive Buttons verwenden, die jeweils beim linken Mausklick ein Event erzeugen sollen. Innerhalb des Bildschirms flitzen \texttt{NOFSTARTPARTICLES} viele Partikel durch die Gegend. Mit den Buttons \texttt{Stopp} und \texttt{Start} sollen die Partikel anhalten bzw. wieder flitzen. 

\myebild{event01.png}{0.7}{Selbst erstellte Events}{picEvent01}

Als weiteres Feature ist so eine Art Zählwerk implementiert. Die Kästchen in der Mitte absorbieren die Partikel und zählen Sie dabei. Die Logik ist wie folgt: Jedes mal, wenn ein Partikel eine Box trifft, wird ein Zählereignis ausgelöst. Dabei wird immer in der Box ganz rechts eine~1 aufaddiert.

Hat die ganz rechte Box den Wert 10 erreicht, erzeugt er einen Überlauf auf die nächste Ziffer links von ihm und setzt sich wieder auf~0; dies setzt sich von rechts nach links fort. Somit wird in den Boxen die Gesamtanzahl von Partikeln angezeigt, die schon verschluckt wurden.

Das Ganze nun im Detail. In der Konsolenausgabe oben (siehe Seite~\pageref{srcConsole00a}) ist für jedes Event so eine eindeutige Nummer zu sehen, über die man das Event identifizieren kann. Pygame reserviert mir einen Nummernbereich für eigene Events zwischen den Konstanten \texttt{pygame.USEREVENT}\randnotiz{USEREVENT}\myindex{pyg}{\texttt{USEREVENT}|underline} und \texttt{pygame.NUMEVENTS~-~1}\randnotiz{NUMEVENTS}\myindex{pyg}{\texttt{NUMEVENTS}|underline}. Für jedes selbst erstellte Event muss man nun eine solche eindeutige Nummer vergeben. Am einfachsten ist es, zentral diese durch \texttt{USEREVENT~+~}$n$ zu definieren. In \zeiref{srcEvent0101} und \zeiref{srcEvent0102} finden Sie entsprechende Beispiele. Ich kapsle diese Definitionen in eine statische Klasse aus keinem anderen Grund, als dass ich dann die Autovervollständigung des Editors gut nutzen kann (\zeiref{srcEvent0100}). Die Klasse \texttt{Settings} sollte selbsterklärend sein.

\lstsource{SRC/00 Einführung/15 Events/events01.py}{1}{22}{python}{Events (2) -- Präambel}{srcEvent01a}

Die Klasse \texttt{Button} sollte auch über weite Strecken verstanden werden. Die erste spannende Stelle finden Sie in \zeiref{srcEvent0103}. Hier wird ein neues \texttt{pygame.event.Event}\randnotiz{Event}-Objekt\myindex{pyg}{\texttt{event}!\texttt{Event}|underline} erzeugt. Als ersten Parameter muss diese eben erwähnte ID angegeben werden. Danach können Sie beliebig viele Angaben als Eventinfo mitgeben. In unserem Beispiel wird der Buttontext mitgegeben, damit man später feststellen kann, welcher Button gedrückt wurde. 

Danach wird in \zeiref{srcEvent0104} über \texttt{pygame.event.post()}\randnotiz{post()}\myindex{pyg}{\texttt{event}!\texttt{post()}|underline} das Event abgefeuert.

\lstsource{SRC/00 Einführung/15 Events/events01.py}{24}{41}{python}{Events (2) -- Klasse \texttt{Button}}{srcEvent01b}

Die Klasse \texttt{Particle} ist viel Quelltext mit wenig Neuem. Partikel zufälliger Größe, Farbe, Richtung und Geschwindigkeit sausen durch den Bildschirm und prallen ggf. von den Rändern ab. Sie enthalten keine event-spezifischen Funktionalitäten. Das Attribut \texttt{\_halted} wird verwendet, um nach Drücken der Buttons den Partikel anzuhalten bzw. wieder loslaufen zu lassen.

\lstsource{SRC/00 Einführung/15 Events/events01.py}{44}{81}{python}{Events (2) -- Klasse \texttt{Particle}}{srcEvent01c}

Mit \texttt{Box} wird so eine Ziffernbox implementiert. Dem Konstruktor wird dabei eine Position und ein Index mitgegeben. Die Bedeutung des Paramters \texttt{position} sollte klar sein. Mit Hilfe von \texttt{index} kann später ermittelt werden, welche Box einen Überlauf zur nächst höheren Zehnerpotenz hatte.

In \texttt{update()} wird der internen Zähler \texttt{\_count} immer um~1 erhöht. Erreiche ich dabei die~10 (\zeiref{srcEvent0105}), wird das Event erzeugt und der Index als Eventinfo übergeben. Damit kann das Hauptprogramm ermitteln, welcher Box er nun ein \texttt{update()} verpassen muss.

\lstsource{SRC/00 Einführung/15 Events/events01.py}{84}{109}{python}{Events (2) -- Klasse \texttt{Box}}{srcEvent01d}

Und jetzt das Hauptprogramm: Im Konstruktor werden die Buttons, Boxen und Partikel angelegt und Spritegroups zugeordnet.

\lstsource{SRC/00 Einführung/15 Events/events01.py}{112}{128}{python}{Events (2) --  Konstruktor von \texttt{Game}}{srcEvent01e}

Die Methode \texttt{run()} ist nahezu langweilig.

\lstsource{SRC/00 Einführung/15 Events/events01.py}{130}{140}{python}{Events (2) --  \texttt{Game.run()}}{srcEvent01f}

Benutzerdefiniert Events werden genauso behandelt wie vordefiniert. Zuerst erfragt man den \texttt{type} und dann verarbeitet man die Eventinfos. In \zeiref{srcEvent0106} wird nachgeschaut, ob einer der beiden Buttons gedrückt wird. Anschließend wird über die Eventinfo \texttt{text} an die Partikel die Nachricht weitergeleitet, ob sie stehenbleiben oder weiterlaufen sollen. Analoges ab \zeiref{srcEvent0107}. Zuerst wird überprüft, ob eine Box einen Überlauf hatte und dann wird mit Hilfe des Eventinfo \texttt{index} die nächste Box darüber informiert, dass sie sich um~1 erhöhen muss.

\lstsource{SRC/00 Einführung/15 Events/events01.py}{142}{156}{python}{Events (2) --  \texttt{Game.watch\_for\_events()}}{srcEvent01g}

Der Rest wird hier der Vollständigkeit abgedruckt.

\lstsource{SRC/00 Einführung/15 Events/events01.py}{158}{999}{python}{Events (2) --  Der Rest von \texttt{Game}}{srcEvent01h}

\subsection{Wie kann ich periodisch Ereignisse erzeugen lassen?}

Dies ist sogar recht einfach. Das vorherige Beispiel wird so erweitert, dass im Abstand von $500~ms$ immer neue Partikel erstellt werden.

Dazu wird zunächst die neue ID \texttt{NEWPARTICLES} für das Benutzerevent definiert.

\lstsource{SRC/00 Einführung/15 Events/events02.py}{10}{13}{python}{Events (3) --  Präambel}{srcEvent02a}

Im Konstruktor von \texttt{Game} wird in \zeiref{srcEvent0201} mit Hilfe von \texttt{pygame.time.set\_timer()} ein periodischer Timer\randnotiz{set\_timer()}\myindex{pyg}{\texttt{time}!\texttt{set\_timer()}|underline} dazu gesetzt. Dieser schießt alle~$500~ms$ die entsprechende EventID ab.

\lstsource{SRC/00 Einführung/15 Events/events02.py}{142}{144}{python}{Events (3) -- Timer}{srcEvent02b}

Wie die anderen Events wird dieses nun in \texttt{watch\_for\_event()} abgefangen (\zeiref{srcEvent0202}) und verarbeitet. Hier indem die Methode \texttt{\_generate\_particles()} aufgerufen wird.

\lstsource{SRC/00 Einführung/15 Events/events02.py}{170}{174}{python}{Events (3) -- Event}{srcEvent02c}

\subsection*{Was war neu?}

Der Vorteil von benutzerdefinierten Ereignissen wird hier gut deutlich. Wollte man dies anders implementieren, müssen die Objekte von einander wissen. Alle Boxen müssten ihren Vorgänger oder Nachfolger beispielsweise als Referenz kennen, um den Überlauf bekannt zu geben. Dies kann auch eine gute Methode sein, durch ein Event werden die Klassen aber entkoppelt und das Hauptprogramm kann die Informationsweiterleitung durch die Eventinfo gesteuert organisierten.

Besonders das Klicken auf die Buttons können durch die Events einfach implementiert werden.


Es wurden folgende Pygame-Elemente eingeführt:

\begin{itemize}
	\item \texttt{USEREVENT}
    \myindex{pyg}{\texttt{USEREVENT}}:\\
    \url{https://www.pygame.org/docs/ref/event.html#pygame.event}

    \item \texttt{NUMEVENTS}
    \myindex{pyg}{\texttt{NUMEVENTS}}:\\
    \url{https://www.pygame.org/docs/ref/event.html#pygame.event}

	\item \texttt{pygame.event.Event}:
	\myindex{pyg}{\texttt{event}!\texttt{Event}}\\
	\url{https://www.pygame.org/docs/ref/event.html#pygame.event.Event}
	
	\item \texttt{pygame.event.get()}
    \myindex{pyg}{\texttt{event!get()}}:\\
    \url{https://www.pygame.org/docs/ref/event.html#pygame.event.get}

	\item \texttt{pygame.event.post()}:
	\myindex{pyg}{\texttt{event}!\texttt{post()}}\\
	\url{https://www.pygame.org/docs/ref/event.html#pygame.event.post}
	
	\item \texttt{pygame.time.set\_timer()}:
	\myindex{pyg}{\texttt{time}!\texttt{set\_timer()}}\\
	\url{https://www.pygame.org/docs/ref/time.html#pygame.time.set_timer}

	
\end{itemize}


