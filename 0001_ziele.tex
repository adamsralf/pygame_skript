%%%%%%%%%%%%%%%%%%%%%%%%%%%%%%%%%%%%%%%%%%%%%%%%%%%%%%%%%%%%%%%%%%%%%
Dieses Skript ist eine Einführung in die Programmierung zweidimensionaler Spiele mit Hilfe von \Gls{pygame} mit der Programmiersprache \Gls{python}. 

Im ersten Teil werden die wichtigsten Konzepte anhand einfacher Beispiele eingeführt. Im zweiten Teil werden Spielprojekte vollständig durchprogrammiert und Probleme bzw. Techniken der Spieleprogrammierung verdeutlicht.

Es bleibt offen, welche Entwicklungsumgebung verwendet wird; ich verwende Visual Code.

Ab der Version~$0.9$ liegt diesem Skript der Pygame-Fork \emph{Pygame Community Edition} (\href{https://pyga.me/}{\nolinkurl{Pygame-ce}}) zu Grunde. Die Quelltexte werden nicht auf Kompatibilität mit dem ursprünglichen Pygame abgeglichen. Der besseren Lesbarkeit halber werde ich aber immer von Pygame sprechen und nicht zwischen den beiden Varianten unterscheiden.

Das Skript wird sich nur mit Spielen beschäftigen, die unmittelbar auf dem Desktop laufen. Oder anders herum: Es wird kein Game-Server, Webspiel oder Mobile-Spiel implementiert. 

Für eine Rückmeldung bei groben Patzern wäre ich sehr dankbar: \href{mailto:adams@tbs1.de}{\nolinkurl{adams@tbs1.de}}.



