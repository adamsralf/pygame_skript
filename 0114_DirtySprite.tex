\newpage
%%%%%%%%%%%%%%%%%%%%%%%%%%%%%%%%%%%%%%%%%%%%%%%%%%%%%%%%%%%%%%%%%%%%%%%%%%%
\section{Dirty Sprites}\label{secDirtySprites}\index{Dirty Sprite}
Derzeit wird in jedem Frame\index{Frame} der gesamte Bildschirm -- also Hintergrund und Sprites -- neu gezeichnet. Dies ist, besonders wenn sich eigentlich nur wenige Sprites bewegen oder ihr Aussehen verändern, Rechenzeitverschwendung. 

Pygame stellt dafür das Konzept der \emph{Dirty Sprites}\randnotiz{Dirty Sprite} zur Verfügung. Dabei wird über das Flag \texttt{pygame.sprite.DirtySprite.dirty}\randnotiz{self.dirty}\myindex{pyg}{\texttt{sprite}!\texttt{DirtySprite}!\texttt{dirty}} gesteuert, ob das Sprite neu gezeichnet werden muss oder nicht. Auch muss der Sprite irgendwie mitgeteilt werden, was auf die alte Position gezeichnet werden soll, wenn sich sie sich bewegt oder verschwindet; wird doch der Hintergrund eben nicht in jedem Frame neu gezeichnet.

\myebild{dirty00.png}{0.7}{Dirty Sprite -- Demo-Spiel}{picDirty00}


\subsection{Einfaches Beispiel}
Wir haben hier ein kleines, relativ simples Spiel (siehe \abbref[vref]{picDirty00}) ohne besonderen Anspruch an Logik oder Ästhetik mit folgenden Features:
\begin{itemize}
	\item Der Bildschirmhintergrund ist ein Sternenhimmel.
	\item Vor dem Hintergrund erscheinen weiße Quadrate. 
	\item Diese Quadrate werden per Zufall rot.
	\item Nach einer gewissen Zeitspanne werden die Quadrate wieder weiß.
	\item Mausklickt man ein rotes Quadrat, verschwindet es.
	\item Das Spiel beendet sich, wenn alle Quadrate verschwunden sind.
	\item Ziel ist es, die Quadrate in möglichst kurzer Zeit wegzuklicken.
\end{itemize}

Das eigentliche Spiel ist etwas anspruchsvoller. Dabei werden die Quadrate in einer gewissen Reihenfolge die Farbe ändern und die Quadrate müssen in dieser Reihenfolge (Kette) angeklickt werden. Mit jedem Level werden die Ketten länger und die Quadrate kleiner. Aber das wäre für unsere Einführung  nur überflüssiger Ballast.

Zunächst ein paar Worte über das noch nicht umgebaute Spiel. Die Klassen \texttt{Settings} und \texttt{Timer} werde ich nicht mehr kommentieren, da die schon ausführlich besprochen wurden. 

Zunächst die Klasse \texttt{Tile}. Eine sehr simple Klasse, die im Konstruktor ein farbiges Rechteck-Surface erzeugt. In \texttt{Tile.update()} werden die Zustandsänderungen definiert. Soll ein Farbwechsel erfolgen (\texttt{action='switch'}) so kann dies nur passieren, wenn der Status der Kachel den Wert~0 hat, was bedeutet, dass die Kachel noch weiß ist. Nur dann wird ein \texttt{Timer} mit einer Periodendauer von $500~ms$ gestartet, die Farbe gewechselt und der Status auf~1 gesetzt. Dadurch wird markiert, dass nun die Kachel durch anklicken zerstört werden kann.

Wird ein Kill-Signal gesendet (\texttt{action='kill'}), wird die Kachel nur dann gelöscht, wenn der Status den Wert~1 hat, also rot eingefärbt ist. Ansonsten wird der Timer überprüft und ggf. Status und Farbe wieder zurückgesetzt.

Erwähnenswert ist noch, dass hier Farbnamen anstelle von RGB-Werten verwendet werden (siehe \zeiref{srcDirty0001} und \zeiref{srcDirty0002}). Dies ist möglich, da in Pygame schon eine große Anzahl von Farben vordefiniert sind\randnotiz{Farbnamen}\index{Farbnamen|underline}. Überall dort, wo ein RGB-Code oder ein Farbwert angegeben werden kann, z.B. im Konstruktor eines \texttt{Color}-Objekts, können diese vordefinierten Farbnamen als Strings angegeben werden.

\lstsource{SRC/00 Einführung/14 Dirty Sprites/dirty00.py}{1}{32}{python}{Dirty Sprite -- \texttt{Tile} (unverändertes Demo)}{srcDirty00a} 

Die Klasse \texttt{Game} ist auch recht einfach. Herzstück ist die Methode \texttt{Game.update()}: Über den Timer \texttt{self.timer} wird in jeder Sekunde eine zufällige Kachel ausgewählt und die Farbe von weiß auf rot gedreht. Ist keine Kachel mehr vorhanden, weil diese nach und nach zerstört wurden, wird das Spiel beendet.

Für unser eigentliches Thema ist die Methode \texttt{Game.draw()} interessant. Hier können Sie sehen, dass in jedem Frame der komplette Hintergrund und alle Kacheln ausgegeben werden, obwohl nur wenige Kacheln ihr Aussehen zwischen zwei Frames verändern bzw. zerstört wurden. Eine offensichtliche Rechenzeitvernichtung. Nehmen wir beispielsweise nur das ständige Zeichnen des Hintergrundes. Bei einer Spielfeldgröße von $800~px\times~400~px$ sind hier $60mal$ in der Sekunde $320.000$~Pixel zu zeichnen, obwohl zwischen zwei Frames eher nur eine Kachel verschwindet, also bei einer Kachelgröße von $30~px\times~30~px$ nur $900$ Pixel neu zu zeichnen wären.

\lstsource{SRC/00 Einführung/14 Dirty Sprites/dirty00.py}{35}{999}{python}{Dirty Sprite -- \texttt{Game} (unverändertes Demo)}{srcDirty00b} 

Aber Rettung naht; wie oben schon erwähnt, wird uns von Pygame die Klasse \texttt{pygame\-.sprite\-.Dirty\-Sprite}\randnotiz{DirtySprite}\myindex{pyg}{\texttt{sprite}!\texttt{DirtySprite}|underline} zur Verfügung gestellt. Diese Klasse wird von \texttt{pygame\-.sprite\-.Sprite} abgeleitet und hat insbesondere ein zusätzliches Attribut, welches steuert, ob der Sprite neu gezeichnet werden muss oder nicht: \texttt{pygame.sprite.DirtySprite.dirty}\randnotiz{self.dirty}\myindex{pyg}{\texttt{sprite}!\texttt{DirtySprite}!\texttt{dirty}|underline}. Dieses Attribut kann drei verschiedene Werte annehmen. In \tabref[vref]{tabDirty} werden die Bedeutungen angegeben. 


Fangen wir also mit dem Umbau an, und machen aus \texttt{Tile} eine Kindklasse von \texttt{Dirty\-Sprite}. In \zeiref{srcDirty0100} wird der Name der Elternklasse angepasst. Im Konstruktor sollte man \texttt{self.dirty} auf~\texttt{1} setzen, damit der Sprite auf jeden Fall beim ersten \texttt{draw()} ausgegeben wird (siehe \zeiref{srcDirty0101})\footnote{Der Wert~1 ist die Vorbelegung für dieses Attribut; ein Setzen ist daher nicht zwingend nötig, aber wegen der besseren Verständlichkeit angebracht.}. Anschließend müssen Sie die Stellen im Quelltext finden, die das Aussehen oder die Position ihres Sprites verändern. Wird eine solche Veränderung vorgenommen, muss \texttt{self.dirty} auf~\texttt{1} gesetzt werden. Dies ist in der Regel die Methode \texttt{update()} oder solche, die von ihr aufgerufen werden. 

In unserem Beispiel wird an zwei Stellen das Aussehen verändert. Zum einen, wenn das Signal \texttt{'switch'} gesendet wird (siehe \zeiref{srcDirty0102}) und zum anderen, wenn der interne Timer die Farbe wieder von rot nach weiß abändert (siehe \zeiref{srcDirty0103}). Wie in \tabref[vref]{tabDirty} beschrieben, wird der Wert automatisch nach der Ausgabe wieder auf~\texttt{0} gesetzt.

Stellt sich noch die Frage, warum nicht auch beim Kill das \texttt{self.dirty} angepasst wird. Wird der Sprite entfernt, soll er ja gerade nicht neu gezeichnet werden; stattdessen soll ja der Hintergrund an dieser Stelle nachgezeichnet werden. Wie das geschieht, wird in \texttt{Game} geregelt.

\lstsource{SRC/00 Einführung/14 Dirty Sprites/dirty01.py}{12}{35}{python}{Dirty Sprite -- \texttt{Tile} (Umbau)}{srcDirty01a} 

Für die Liste der Kacheln, können wir nun kein \texttt{pygame.sprite.Group}-Objekt\myindex{pyg}{\texttt{sprite}!\texttt{Group}} mehr nehmen, da diese Liste keine Attribute und Methoden von \texttt{pygame.sprite.DirtySprite} kennt. Die passende Alternative zur \texttt{Group}-Klasse ist \texttt{pygame.sprite.LayeredDirty}\myindex{pyg}{\texttt{sprite}!\texttt{LayeredDirty}|underline}\randnotiz{LayeredDirty}. Diese Klasse enthält alle Mechanismen, die wir für die Unterstützung von \texttt{DirtySprite} brauchen. Bauen wir daher erstmal das entsprechende Attribut in \zeiref{srcDirty0104} um.

\lstsource{SRC/00 Einführung/14 Dirty Sprites/dirty01.py}{38}{51}{python}{Dirty Sprite -- Konstruktor von \texttt{Game} (Umbau)}{srcDirty01b} 

Würden wir dieses Programm nun ausführen, bekämen wir ein sehr unbefriedigendes Ergebnis zu sehen. Einmal erscheinen ganz kurz (nur für ein Frame) alle Kacheln in weiß und danach ist nur noch der Hintergrund zu sehen. Ab und zu blitzen weiße und rote Kacheln auf, und zwar immer dann, wenn ein Farbwechsel erfolgt, also \texttt{dirty} auf~\texttt{1} gesetzt wurde. Wir müssen hier also noch weitere Veränderungen vornehmen. 

Zunächst müssen wir uns klar machen, dass nun in \zeiref{srcDirty0201} nicht mehr das \texttt{draw()} von \texttt{Group}, sondern von \texttt{LayeredDirty}\myindex{pyg}{\texttt{sprite}!\texttt{LayeredDirty}!\texttt{draw()}|underline}\randnotiz{draw()} aufgerufen wird. Diese Methode kennt nämlich noch einen zweiten Parameter: das Hintergrundbitmap. Verschwindet nun eine Kachel, weil darauf geklickt wurde, wird der passende Ausschnitt aus dem Hintergrundbitmap an Stelle der Kachel ausgegeben. Auch merkt \texttt{LayeredDirty}, dass der Hintergrund vorher noch nicht ausgegeben wurde und blittet ihn beim ersten Aufruf von \texttt{draw()} einmalig komplett. Deshalb kann die Zeile, die in der vorherigen Version den Hintergrund immer blittet, entfallen.

Ein weiterer Unterschied von \texttt{LayeredDirty.draw()} ist, dass es eine Liste von Rechtecken zurückliefert. Und zwar nur von den Rechtecken, die geänderte Bildschirmbareiche markieren. Verwenden wir nun nicht mehr \texttt{pygame.display.flip()}\myindex{pyg}{\texttt{display}!\texttt{flip()}}, sondern \texttt{pygame.display.update()}\myindex{pyg}{\texttt{display}!\texttt{update()}|underline}\randnotiz{update()} (siehe \zeiref{srcDirty0202}), so können wir diese veränderten Bildschirmbereiche als Parameter übergeben und nur diese Bereiche werden dann neu gezeichnet.

\lstsource{SRC/00 Einführung/14 Dirty Sprites/dirty02.py}{62}{64}{python}{Dirty Sprite -- \texttt{Game.draw()} (Umbau)}{srcDirty02a} 


Wenn wir jetzt das Programm ausführen, sieht alles soweit ganz gut aus. Es verbleiben mir aber noch zwei Kleinigkeiten. In \texttt{draw()} wird jetzt bei jedem Aufruf in \zeiref{srcDirty0201} das Backgroundimage mitgeliefert. Das passiert immerhin 60 mal pro Sekunde. Wäre es nicht schöner, wir würden einmal das den Hintergrund setzen und könnten dann auf den zweiten Paramter in der Zeile verzichten? Na klar wäre das schöner und deshalb geht das auch ;-)

In \zeiref{srcDirty0301} wird der Hintergrund mit \texttt{pygame.sprite.LayeredDirty.clear()}\myindex{pyg}{\texttt{sprite}!\texttt{LayeredDirty}!\texttt{clear()}|underline}\randnotiz{clear()} gesetzt. Auch wird festgelegt, auf welches Surface der Hintergrund gezeichnet werden soll, hier eben auf \texttt{self.screen}.

\lstsource{SRC/00 Einführung/14 Dirty Sprites/dirty03.py}{46}{49}{python}{Dirty Sprite -- Konstruktor von \texttt{Game} (Umbau)}{srcDirty03a} 

Deshalb kann in \texttt{draw()} auf den zweiten Parameter verzichtet werden (\zeiref{srcDirty0303}).

\lstsource{SRC/00 Einführung/14 Dirty Sprites/dirty03.py}{64}{66}{python}{Dirty Sprite -- \texttt{Game.draw()} (Umbau)}{srcDirty03b} 

Ich sprach aber von zwei Kleinigkeiten. Für die zweite muss ich ein wenig was erklären. Die ganze Idee um den \texttt{DirtySprite} herum ist ja, Performance dadurch einzusparen, dass man nur noch die veränderten Bildschirmbereiche aktualisiert. Nun kostet aber das Ermitteln und Ausschneiden dieser Bereiche ebenfalls Rechenzeit. In der Informatik spricht man dabei von einem \gls{tradeoff}. Diese Rechenzeit kann das Zeitfenster, welches dafür innerhalb eines Frames zur Verfügung steht, überschreiten und damit die Bildschirmausgabe verlangsamen bzw. qualitativ verschlechtern. Daher wird während der Ausführung von \texttt{draw()} in \texttt{LayeredDirty} die Ausführungszeit gemessen\footnote{Siehe \url{https://github.com/pygame/pygame/blob/main/src_py/sprite.py}.}. Wird das Zeitfenster überschritten, merkt sich das \texttt{LayeredDirty} und blittet beim nächsten mal Hintergründe und Sprites, als ob keine DirtySprite-Logik verwendet wird. 

Aber woher soll nun \texttt{DirtyLayer} wissen, wie lang das verfügbare Zeitfenster ist? Eben dafür ist die Methode \texttt{pygame.sprite.LayeredDirty.set\-\_tim\-ing\-\_tres\-hold()}\myindex{pyg}{\texttt{sprite}!\texttt{LayeredDirty}!\texttt{set\_timing\_treshold()}|underline}\randnotiz{set\-\_tim\-ing\-\_tres\-hold()}\footnote{Der Methodenname enthält einen Rechtschreibfehler und soll im nächsten Pygame-Release auf \texttt{set\_timing\_threshold()} korrigiert werden.} zuständig (siehe \zeiref{srcDirty0302}). Im Handbuch wird vorgeschlagen, den Wert $1000.0 / fps$ zu nehmen. Warum? Eine Sekunde besteht aus $1000$ Millisekunden. Teilt man diese $1000$ durch die Anzahl der Frames pro Sekunde, erhält man die Anzahl der Millisekunden, die ein Frame zur Verfügung hat; bei uns sind es ca.~$16~ms$.

\subsection{Performancemessungen}

Der letzte Absatz im vorherigen Abschnitt hat mich misstrauisch gemacht. Sind \texttt{Dirty\-Sprite}-Objekte wirklich schneller als normale? Und ich muss gestehen, dass ich erst recht erschreckende Ergebnisse bekommen habe. Aber der Reihe nach.

Zunächst habe ich obiges Beispiel ein wenig umgebaut. Die Kacheln werden dabei nicht mit der Maus angeklickt, sondern nach zweimaligem Farbwechsel löschen die sich selbst. Sind dann keine Kacheln mehr da, beendet sich der Testlauf. An der entscheidenden Stelle habe ich dann Zeitwerte abgegriffen, um diese in eine Datei zu schreiben. 

\lstsource{SRC/00 Einführung/14 Dirty Sprites/performance00.py}{71}{84}{python}{Performancevergleich -- Messung}{srcPerformance00a} 

Dann habe ich Testläufe mit den Parametern aus \tabref[vref]{tabPerformance} auf einem Schul-PC laufen lassen. Die Werte waren unterschiedlich\footnote{Die Werte werden hier bereinigt verwendet. So wurden durch Störungen wie beispielsweise eingehende E-Mails kurzfristige Spitzen erzeugt, die rausgerechnet wurden. Ebenso wurden die Anzahl der Messwerte angeglichen.}, aber die Tendenz immer die gleiche. Ich habe mal ein Ergebnis in einer Grafik visualisiert (siehe \abbref[vref]{picDirtyPerformance00}). Das Ergebnis ist eindeutig, DirtySprites sind signifikant schneller.

\begin{longtable}{rcr}
	\caption{Testkonfiguration Performancemassung}\label{tabPerformance} \\
	% Definition des Tabellenkopfes auf der ersten Seite
	Testlauf & Kachelgröße & Anzahl Kacheln \\\hline\hline
	\hline
	\endfirsthead % Erster Kopf zu Ende
	% Definition des Tabellenkopfes auf den folgenden Seiten
	\caption{Testkonfiguration Performancemassung (Fortsetzung)}\\
	Testlauf & Kachelgröße & Anzahl Kacheln \\\hline\hline
	\hline
	\endhead % Zweiter Kopf ist zu Ende
	% Ab hier kommt der Inhalt der Tabelle
	1  &   $5~px\times 5~px$ &  100\\ \hline
	2  &   $5~px\times 5~px$ & 4000\\ \hline
	3  &  $30~px\times 30~px$ &  100 \\ \hline
	4  &  $50~px\times 50~px$ &  100\\ \hline
	5  & $100~px\times 100~px$ &   40\\ \hline
\end{longtable} 


\myebild{tbs1_05_4000.pdf}{1.0}{Performancevergleich mit Testkonfiguration~2}{picDirtyPerformance00}

Doch leider meinte ich, das Experiment zu Hause wiederholen zu müssen, da das Programm noch ein wenig nachoptimiert wurde. Und dann -- oh Schreck -- kam \abbref[vref]{picDirtyPerformance01} raus. Natürlich fieberhaft nach einem Fehler gesucht, schließlich hatte ich ja Kleinigkeiten verändert. Test auf einem Surface~Pro wiederholt und schon wieder recht schlechte Ergebnisse (\abbref[vref]{picDirtyPerformance02}). Hier war es sogar noch verwirrender, da die ersten rund 8.000~Iterationen einen Vorteil für DirtySprites ergaben und danach die Zeitverbräuche sich angleichten. 

Nach vielen Wiederholungstest auf den drei Rechnern, kam mir der Gedanke, dass der PC zu Hause und der Surface~Pro vielleicht nicht die Framerate von $60~fps$ schaffen und daher das Zeitfenster zu klein ist. Also den Test mit kleinerer Framerate wiederholt und siehe da, dann waren die Ergebnisse wieder eindeutig (siehe \abbref[vref]{picDirtyPerformance03}). Ein Performancetest zwischen den einzelnen Rechnern bestätigte im Nachgang diese Vermutung. In \abbref[vref]{picDirtyPerformance04} sehen Sie, das der Schul-PC definitiv die beste Grafikverarbeitungsgeschwindigkeit hat.

\myezweihbild{zuhause_05_4000.pdf}{0.55}{Testkonfiguration~2 mit privatem PC}{picDirtyPerformance01}{surface_05_4000.pdf}{0.55}{Testkonfiguration~2 mit Surface Pro}{picDirtyPerformance02}
%\myebild{zuhause_05_4000.pdf}{1.0}{Testkonfiguration~2 auf einem privatem PC}{picDirtyPerformance01}
%\myebild{surface_05_4000.pdf}{1.0}{Testkonfiguration~2 auf einem Surface Pro}{picDirtyPerformance02}

\myezweihbild{fps_05_4000.pdf}{0.55}{Testkonfiguration~2 mit privatem PC und reduzierter fps}{picDirtyPerformance03}{ohne_05_100.pdf}{0.55}{Bestätigung der Unterschiede}{picDirtyPerformance04}
%\myebild{fps_05_4000.pdf}{1.0}{Testkonfiguration~2 auf einem privatem PC mit reduzierter fps}{picDirtyPerformance03}

%\myebild{ohne_05_100.pdf}{1.0}{Bestätigung der Rechenleistungsunterschiede der Testrechner}{picDirtyPerformance04}

Es ist somit eine ernst zu nehmende Aufgabe, die maximale Framerate\index{Framerate} auf dem Rechner zu ermitteln. Nur dann können die Mechanismen des DirtySprite-Konzepts greifen. Auch ist nicht auszuschließen, dass die Rechner deshalb kein schönes ruckelfreies Bewegungsbild erzeugen, wenn die Framerate für den Rechner zu hoch eingestellt ist.

\subsection*{Was war neu?}

Mit Hilfe von entsprechenden Klassen, kann die Zeichenausgabe pro Frame auf die Bereiche eingeschränkt werden, die sich tatsächlich verändert haben. Der Performanceverbrauch reduziert sich dabei erheblich. 

Es muss allerdings darauf geachtet werden, dass die Framerate der Leistungsfähigkeit der Rechnerkonfiguration angepasst wird. 

Es wurden folgende Pygame-Elemente eingeführt:
\begin{itemize}
	\item Vordefnierte Farbnamen:\index{Farbennamen}\\ 
	\url{http://www.pygame.org/docs/ref/color_list.html}

	\item\texttt{pygame.display.update()}:
	\myindex{pyg}{\texttt{display}!\texttt{update()}}\\ \url{http://www.pygame.org/docs/ref/sprite.html#pygame.display.update}

	\item\texttt{pygame.sprite.DirtySprite}:
	\myindex{pyg}{\texttt{sprite}!\texttt{DirtySprite}}\\ \url{http://www.pygame.org/docs/ref/sprite.html#pygame.sprite.DirtySprite}

	\item\texttt{pygame.sprite.DirtySprite.dirty}:
	\myindex{pyg}{\texttt{sprite}!\texttt{DirtySprite}!\texttt{dirty}}\\ \url{http://www.pygame.org/docs/ref/sprite.html#pygame.sprite.DirtySprite}\\
	Bedeutung siehe \tabref[vref]{tabDirty}.

	\item \texttt{pygame.sprite.LayeredDirty}:
	\myindex{pyg}{\texttt{sprite}!\texttt{LayeredDirty}}\\ \url{http://www.pygame.org/docs/ref/sprite.html#pygame.sprite.LayeredDirty}
	
	\item \texttt{pygame.sprite.LayeredDirty.clear()}:
	\myindex{pyg}{\texttt{sprite}!\texttt{LayeredDirty}!\texttt{clear()}}\\ \url{http://www.pygame.org/docs/ref/sprite.html#pygame.sprite.LayeredDirty.clear}
	
	\item \texttt{pygame.sprite.LayeredDirty.draw()}:
	\myindex{pyg}{\texttt{sprite}!\texttt{LayeredDirty}!\texttt{draw()}}\\ \url{http://www.pygame.org/docs/ref/sprite.html#pygame.sprite.LayeredDirty.draw}

	\item \texttt{pygame.sprite.LayeredDirty.set\_timing\_treshold()}:
	\myindex{pyg}{\texttt{sprite}!\texttt{LayeredDirty}!\texttt{set\_timing\_treshold()}}\\ \url{http://www.pygame.org/docs/ref/sprite.html#pygame.sprite.LayeredDirty.set\_timing\_treshold}

	\item \texttt{pygame.sprite.LayeredDirty.set\_timing\_threshold()}:
	\myindex{pyg}{\texttt{sprite}!\texttt{LayeredDirty}!\texttt{set\_timing\_threshold()}}\\ \url{http://www.pygame.org/docs/ref/sprite.html#pygame.sprite.LayeredDirty.set\_timing\_threshold}

\end{itemize}

	\begin{longtable}{lp{12cm}}
	\caption{Bedeutung von \texttt{dirty}}\label{tabDirty} \\
	% Definition des Tabellenkopfes auf der ersten Seite
	Konstante & Beschreibung \\\hline\hline
	\hline
	\endfirsthead % Erster Kopf zu Ende
	% Definition des Tabellenkopfes auf den folgenden Seiten
	\caption{Bedeutung von \texttt{dirty} (Fortsetzung)}\\
	Konstante & Beschreibung \\\hline\hline
	\hline
	\endhead % Zweiter Kopf ist zu Ende
	% Ab hier kommt der Inhalt der Tabelle
	\texttt{0}  & Der Sprite ist noch aktuell und muss nicht neu gezeichnet werden.\\ \hline
	\texttt{1}  & (Default) Der Sprite ist veraltet (Aussehen oder Position haben sich verändert) und muss neu gezeichnet werden. Nach dem Neuzeichnen wird \texttt{dirty} automatisch wieder auf~\texttt{0} gesetzt. \\ \hline
	\texttt{2}  & Der Sprite wird immer neu gezeichnet. Sie wird nicht nach dem Zeichnen auf \texttt{0} gesetzt.\\ \hline
\end{longtable} 
