\newpage
%%%%%%%%%%%%%%%%%%%%%%%%%%%%%%%%%%%%%%%%%%%%%%%%%%%%%%%%%%%%%%%%%%%%%%%%%%%
\section{Tastatur}\index{Tastatur}\label{secTastatur}
%\subsection{Beispiele}

%\begin{wrapfigure}[10]{r}{3.1cm}%
%	\begin{center}%
%		\vspace{-1cm}%
%		\myfigure{invader07.png}{0.8}{Tastatur}{picInvader07}%
%	\end{center}%
%\end{wrapfigure}%
Ich möchte hier die Tastatur nicht erschöpfend behandeln, sondern lediglich das Grundprinzip verdeutlichen. So soll die Bewegungsrichtung durch die Pfeiltasten gesteuert werden können. Ebenso soll das Raumschiffe stehen bleiben oder sich wieder in Bewegung setzen können. Auch kann das Spiel jetzt durch die Escape-Taste verlassen werden (\Gls{boss}).

Im ersten Schritt wird ein Dictionary\index{Dictionary} der möglichen Richtungen in \texttt{Settings} angelegt. Diese werden als \texttt{Vector2}-Objekte\myindex{pyg}{\texttt{math}!\texttt{Vector2D}} verwaltet, da diese einfacher für mathematische Operationen verwendet werden können.

\lstsource{SRC/00 Einführung/06 Tastatur/tastatur00.py}{7}{16}{python}{Bewegung durch Tastatur steuern (1), \texttt{Settings}}{srcTastatur00aa} 


Danach bereiten wir die Verteidiger-Klasse vor bzw. wandeln sie ein wenig ab (\srcref[vref]{srcTastatur00a}). Das Sprite wird nun nicht mehr unten sondern mittig platziert (\zeiref{srcTastatur0001}), und das Raumschiff soll sich jetzt auch vertikal bewegen können. Dazu braucht es entweder zwei entsprechende Variablen oder aber ein \texttt{Vector2}-Objekt. Ich nehme ein \texttt{Vector2}-Objekt (\zeiref{srcTastatur0002}), wobei das erste Element der Richtungsvektor der horizontalen und das zweite der vertikalen Richtung ist. Der jeweilige Richtungsvektor wird dabei entsprechend der schon oben vorgestellten Semantik gesetzt. In der Methode \texttt{change\_direction()} wird entweder der Richtungsvektor gesetzt oder die Geschwindigkeit angepasst. Bewegen und Stehenbleiben wird einfach dadurch erreicht, dass ich die Geschwindigkeit bei \texttt{start} auf~100 bzw. bei \texttt{stop} auf~0 setze.

\lstsource{SRC/00 Einführung/06 Tastatur/tastatur00.py}{19}{54}{python}{Bewegung durch Tastatur steuern (1), \texttt{Defender}}{srcTastatur00a} 

Kommen wir jetzt zur eigentlichen Tastaturverarbeitung: Das Verwenden einer Taste kann die Ereignistypen\index{Ereignis}\myindex{pyg}{\texttt{Event}} \texttt{pygame.KEYDOWN}\myindex{pyg}{\texttt{KEYDOWN}|underline} oder \texttt{pygame.KEYUP}\myindex{pyg}{\texttt{KEYUP}|underline}\randnotiz{KEYDOWN\\KEYUP} auslösen. In unserem Beispiel (\zeiref{srcTastatur0003}) wollen wir wissen, welche Taste \emph{gedrückt} wurde, also verwenden wir \texttt{KEYDOWN}. Anschließend können wir über \texttt{pygame.event.key}\myindex{pyg}{\texttt{event}!\texttt{key}}\randnotiz{key} ermitteln, welche Taste gedrückt wurde. Dazu stellt uns Pygame in \texttt{pygame.key}\myindex{pyg}{\texttt{key}} eine Liste von vordefinierten Konstanten zur Verfügung (siehe \tabref[vref]{tabKey} und \tabref[vref]{tabKeyMod}).

\lstsource{SRC/00 Einführung/06 Tastatur/tastatur00.py}{80}{101}{python}{Bewegung durch Tastatur steuern (4), \texttt{Game}.\texttt{watch\_for\_events()}}{srcTastatur00d}

Fangen wir mit der Boss-Taste an. In \zeiref{srcTastatur0004} wird über die Konstante \texttt{K\_ESCAPE}\myindex{pyg}{\texttt{K\_ESCAPE}}\randnotiz{K\_ESCAPE} abgefragt, ob die gedrückte Taste die Escape-Taste ist. Wie beim \emph{Weg-Xen} wird danach einfach das Flag der Hauptprogrammschleife auf \false\ gesetzt. Probieren Sie es aus!

Danach werden mit Hilfe von \randnotiz{K\_LEFT K\_RIGHT K\_UP K\_DOWN}\texttt{K\_LEFT}, \texttt{K\_RIGHT}, \texttt{K\_UP} und \texttt{K\_DOWN}\myindex{pyg}{\texttt{K\_LEFT}}\myindex{pyg}{\texttt{K\_RIGHT}}\myindex{pyg}{\texttt{K\_UP}}\myindex{pyg}{\texttt{K\_DOWN}} ab \zeiref{srcTastatur0005}ff. die vier Pfeiltasten abgefragt und die entsprechende Nachricht an den Verteidiger gesendet.

Mit Hilfe der Leerzeichen-Taste \texttt{K\_SPACE}\myindex{pyg}{\texttt{K\_SPACE}}\randnotiz{K\_SPACE} wird das Raumschiff in \zeiref{srcTastatur0006} gestoppt. 

Um den Einsatz der Shift-Taste (Umschalttaste)\index{Shift-Taste} mal zu demonstrieren, habe ich hier das~\texttt{r} doppelt belegt (\zeiref{srcTastatur0007}). Das große~\texttt{R} stoppt das Raumschiff und das kleine~\texttt{r} startet es wieder. Dabei wird die Variable \texttt{event.mod}\myindex{pyg}{\texttt{event}!\texttt{mod}}\randnotiz{event.mod} mit Hilfe einer bitweisen Und-Verknüpfung dahingehend überprüft, ob das entsprechende Bit \texttt{KMOD\_LSHIFT}\myindex{pyg}{\texttt{KMOD\_LSHIFT}}\randnotiz{KMOD\_LSHIFT} für die linke Shift-Taste gedrückt wurde.

Dies soll ersteinmal ausreichen. Die Tastatur ist nur eine Möglichkeit der Spielsteuerung. Maus, Game-Controller oder Joystick sind ebenfalls in Pygame möglich.

\subsection*{Was war neu?}
Die Tastatur sendet Ereignisnachrichten, die man abfangen und auswerten kann. Dabei wird zum einen unterschieden, was mit der Tastatur gemacht wurde (\texttt{event.typ}) und dann mit welcher Taste (\texttt{event.key}). Über \texttt{event.mod} kann bitweise abgefragt werden, welche Steuertasten auf der Tastatur verwendet wurden.

Es wurden folgende Pygame-Elemente eingeführt:


\begin{itemize}
	\item \texttt{pygame.key}:
	\myindex{pyg}{\texttt{KEY}|underline}\\ \url{https://pyga.me/docs/ref/key.html}

	\item \texttt{pygame.KEYDOWN}, \texttt{pygame.KEYUP}:
	\myindex{pyg}{\texttt{KEYDOWN}}\myindex{pyg}{\texttt{KEYUP}}\\ \url{https://pyga.me/docs/ref/event.html}
	
\end{itemize}

\begin{longtable}{lll}
	\caption{Liste von vordefinierten Tastaturkonstanten}\label{tabKey}\index{Tastatur!-konstanten} \\
	% Definition des Tabellenkopfes auf der ersten Seite
     Konstante & Bedeutung & Beschreibung \\\hline\hline
	\hline
	\endfirsthead % Erster Kopf zu Ende
	% Definition des Tabellenkopfes auf den folgenden Seiten
	\caption{Liste von vordefinierten Tastaturkonstanten (Fortsetzung)}\\
	 Konstante & Bedeutung & Beschreibung \\\hline\hline
	\hline
	\endhead % Zweiter Kopf ist zu Ende
	% Ab hier kommt der Inhalt der Tabelle
\texttt{K\_BACKSPACE}    &  \verb+\b+    &  Löschen (backspace) \\ \hline
\texttt{K\_TAB}          &  \verb+\t+    &  Tabulator\\ \hline
\texttt{K\_CLEAR}        &  \verb++      &  Leeren\\ \hline
\texttt{K\_RETURN}       &  \verb+\r+    &  Eingabe (return, enter)\\ \hline
\texttt{K\_PAUSE}        &  \verb++      &  Pause\\ \hline
\texttt{K\_ESCAPE}       &  \verb+^[+    &  Abbruch (escape)\\ \hline
\texttt{K\_SPACE}        &  \verb+ +     &  Leerzeichen (space)\\ \hline
\texttt{K\_EXCLAIM}      &  \verb+!+     &  Ausrufezeichen\\ \hline
\texttt{K\_QUOTEDBL}     &  \verb+"+     &  Gänsefüßchen\\ \hline
\texttt{K\_HASH}         &  \verb+#+     &  Doppelkreuz (hash)\\ \hline
\texttt{K\_DOLLAR}       &  \verb+$+     &  Dollar\\ \hline
\texttt{K\_AMPERSAND}    &  \verb+&+     &  Kaufmannsund\\ \hline
\texttt{K\_QUOTE}        &  \verb+'+     &  Hochkomma\\ \hline
\texttt{K\_LEFTPAREN}    &  \verb+(+     &  Linke runde Klammer\\ \hline
\texttt{K\_RIGHTPAREN}   &  \verb+)+     &  Rechte runde Klammer\\ \hline
\texttt{K\_ASTERISK}     &  \verb+*+     &  Sternchen\\ \hline
\texttt{K\_PLUS}         &  \verb-+-     &  Plus\\ \hline
\texttt{K\_COMMA}        &  \verb+,+     &  Komma\\ \hline
\texttt{K\_MINUS}        &  \verb+-+     &  Minus\\ \hline
\texttt{K\_PERIOD}       &  \verb+.+     &  Punkt\\ \hline
\texttt{K\_SLASH}        &  \verb+/+     &  Schrägstrich\\ \hline
\texttt{K\_0}            &  \verb+0+     &  0\\ \hline
\texttt{K\_1}            &  \verb+1+     &  1\\ \hline
\texttt{K\_2}            &  \verb+2+     &  2\\ \hline
\texttt{K\_3}            &  \verb+3+     &  3\\ \hline
\texttt{K\_4}            &  \verb+4+     &  4\\ \hline
\texttt{K\_5}            &  \verb+5+     &  5\\ \hline
\texttt{K\_6}            &  \verb+6+     &  6\\ \hline
\texttt{K\_7}            &  \verb+7+     &  7\\ \hline
\texttt{K\_8}            &  \verb+8+     &  8\\ \hline
\texttt{K\_9}            &  \verb+9+     &  9\\ \hline
\texttt{K\_COLON}        &  \verb+:+     &  Doppelpunkt\\ \hline
\texttt{K\_SEMICOLON}    &  \verb+;+     &  Semicolon\\ \hline
\texttt{K\_LESS}         &  \verb+<+     &  Kleiner\\ \hline
\texttt{K\_EQUALS}       &  \verb+=+     &  Gleich\\ \hline
\texttt{K\_GREATER}      &  \verb+>+     &  Größer\\ \hline
\texttt{K\_QUESTION}     &  \verb+?+     &  Fragezeichen\\ \hline
\texttt{K\_AT}           &  \makeatletter \verb+@+ \makeatother    &  Klammeraffe\\ \hline
\texttt{K\_LEFTBRACKET}  &  \verb+[+     &  Linke eckige Klammer\\ \hline
\texttt{K\_BACKSLASH}    &  \verb+\+     &  Umgekehrter Schrägstrich\\ \hline
\texttt{K\_RIGHTBRACKET} &  \verb+]+     &  Rechte eckige Klammer\\ \hline
\texttt{K\_CARET}        &  \verb+^+     &  Hütchen\\ \hline
\texttt{K\_UNDERSCORE}   &  \verb+_+     &  Unterstrich\\ \hline
\texttt{K\_BACKQUOTE}    &  \verb+`+     &  Akzent Grvis\\ \hline
\texttt{K\_a}            &  \verb+a+     &  a\\ \hline
\texttt{K\_b}            &  \verb+b+     &  b\\ \hline
\texttt{K\_c}            &  \verb+c+     &  c\\ \hline
\texttt{K\_d}            &  \verb+d+     &  d\\ \hline
\texttt{K\_e}            &  \verb+e+     &  e\\ \hline
\texttt{K\_f}            &  \verb+f+     &  f\\ \hline
\texttt{K\_g}            &  \verb+g+     &  g\\ \hline
\texttt{K\_h}            &  \verb+h+     &  h\\ \hline
\texttt{K\_i}            &  \verb+i+     &  i\\ \hline
\texttt{K\_j}            &  \verb+j+     &  j\\ \hline
\texttt{K\_k}            &  \verb+k+     &  k\\ \hline
\texttt{K\_l}            &  \verb+l+     &  l\\ \hline
\texttt{K\_m}            &  \verb+m+     &  m\\ \hline
\texttt{K\_n}            &  \verb+n+     &  n\\ \hline
\texttt{K\_o}            &  \verb+o+     &  o\\ \hline
\texttt{K\_p}            &  \verb+p+     &  p\\ \hline
\texttt{K\_q}            &  \verb+q+     &  q\\ \hline
\texttt{K\_r}            &  \verb+r+     &  r\\ \hline
\texttt{K\_s}            &  \verb+s+     &  s\\ \hline
\texttt{K\_t}            &  \verb+t+     &  t\\ \hline
\texttt{K\_u}            &  \verb+u+     &  u\\ \hline
\texttt{K\_v}            &  \verb+v+     &  v\\ \hline
\texttt{K\_w}            &  \verb+w+     &  w\\ \hline
\texttt{K\_x}            &  \verb+x+     &  x\\ \hline
\texttt{K\_y}            &  \verb+y+     &  y\\ \hline
\texttt{K\_z}            &  \verb+z+     &  z\\ \hline
\texttt{K\_DELETE}       &  \verb+ +     &  Löschen (delete)\\ \hline
\texttt{K\_KP0}          &  \verb+ +     &  Nummernfeld 0\\ \hline
\texttt{K\_KP1}          &  \verb+ +     &  Nummernfeld 1\\ \hline
\texttt{K\_KP2}          &  \verb+ +     &  Nummernfeld 2\\ \hline
\texttt{K\_KP3}          &  \verb+ +     &  Nummernfeld 3\\ \hline
\texttt{K\_KP4}          &  \verb+ +     &  Nummernfeld 4\\ \hline
\texttt{K\_KP5}          &  \verb+ +     &  Nummernfeld 5\\ \hline
\texttt{K\_KP6}          &  \verb+ +     &  Nummernfeld 6\\ \hline
\texttt{K\_KP7}          &  \verb+ +     &  Nummernfeld 7\\ \hline
\texttt{K\_KP8}          &  \verb+ +     &  Nummernfeld 8\\ \hline
\texttt{K\_KP9}          &  \verb+ +     &  Nummernfeld 9\\ \hline
\texttt{K\_KP\_PERIOD}   &  \verb+.+     &  Nummernfeld Punkt\\ \hline
\texttt{K\_KP\_DIVIDE}   &  \verb+/+     &  Nummernfeld Geteilt/Schrägstrich\\ \hline
\texttt{K\_KP\_MULTIPLY} &  \verb+*+     &  Nummernfeld Mal/Sternchen\\ \hline
\texttt{K\_KP\_MINUS}    &  \verb+-+     &  Nummernfeld Minus\\ \hline
\texttt{K\_KP\_PLUS}     &  \verb-+-     &  Nummernfeld Plus\\ \hline
\texttt{K\_KP\_ENTER}    &  \verb+\r+    &  Nummernfeld Eingabe (return, enter)\\ \hline
\texttt{K\_KP\_EQUALS}   &  \verb+=+     &  Nummernfeld Gleich\\ \hline
\texttt{K\_UP}           &  \verb+ +     &  Pfeil nach oben\\ \hline
\texttt{K\_DOWN}         &  \verb+ +     &  Pfeil nach unten\\ \hline
\texttt{K\_RIGHT}        &  \verb+ +     &  Pfeil nach rechts\\ \hline
\texttt{K\_LEFT}         &  \verb+ +     &  Pfeil nach links\\ \hline
\texttt{K\_INSERT}       &  \verb+ +     &  Einfügen ein/aus\\ \hline
\texttt{K\_HOME}         &  \verb+ +     &  Pos1\\ \hline
\texttt{K\_END}          &  \verb+ +     &  Ende\\ \hline
\texttt{K\_PAGEUP}       &  \verb+ +     &  Hochblättern\\ \hline
\texttt{K\_PAGEDOWN}     &  \verb+ +     &  Runterblättern\\ \hline
\texttt{K\_F1}           &  \verb+ +     &  F1\\ \hline
\texttt{K\_F2}           &  \verb+ +     &  F2\\ \hline
\texttt{K\_F3}           &  \verb+ +     &  F3\\ \hline
\texttt{K\_F4}           &  \verb+ +     &  F4\\ \hline
\texttt{K\_F5}           &  \verb+ +     &  F5\\ \hline
\texttt{K\_F6}           &  \verb+ +     &  F6\\ \hline
\texttt{K\_F7}           &  \verb+ +     &  F7\\ \hline
\texttt{K\_F8}           &  \verb+ +     &  F8\\ \hline
\texttt{K\_F9}           &  \verb+ +     &  F9\\ \hline
\texttt{K\_F10}          &  \verb+ +     &  F10\\ \hline
\texttt{K\_F11}          &  \verb+ +     &  F11\\ \hline
\texttt{K\_F12}          &  \verb+ +     &  F12\\ \hline
\texttt{K\_F13}          &  \verb+ +     &  F13\\ \hline
\texttt{K\_F14}          &  \verb+ +     &  F14\\ \hline
\texttt{K\_F15}          &  \verb+ +     &  F15\\ \hline
\texttt{K\_NUMLOCK}      &  \verb+ +     &  Umschalten Zahlen\\ \hline
\texttt{K\_CAPSLOCK}     &  \verb+ +     &  Umschalten Großbuchstaben\\ \hline
\texttt{K\_SCROLLOCK}    &  \verb+ +     &  Umschalten auf scrollen\\ \hline
\texttt{K\_RSHIFT}       &  \verb+ +     &  Rechte Umschalttaste\\ \hline
\texttt{K\_LSHIFT}       &  \verb+ +     &  Linke Umschalttaste\\ \hline
\texttt{K\_RCTRL}        &  \verb+ +     &  Rechte Steuerungstaste\\ \hline
\texttt{K\_LCTRL}        &  \verb+ +     &  Linke Steuerungstaste\\ \hline
\texttt{K\_RALT}         &  \verb+ +     &  Rechte Alterntivtaste\\ \hline
\texttt{K\_LALT}         &  \verb+ +     &  Linke Alternativtaste\\ \hline
\texttt{K\_RMETA}        &  \verb+ +     &  Rechte Metataste\\ \hline
\texttt{K\_LMETA}        &  \verb+ +     &  Linke Metataste\\ \hline
\texttt{K\_LSUPER}       &  \verb+ +     &  Linke Windowstaste\\ \hline
\texttt{K\_RSUPER}       &  \verb+ +     &  Rechte Windowstaste\\ \hline
\texttt{K\_MODE}         &  \verb+ +     &  AltGr Umschalter\\ \hline
\texttt{K\_HELP}         &  \verb+ +     &  Hilfe\\ \hline
\texttt{K\_PRINT}        &  \verb+ +     &  Bildschirmdruck/Screenshot\\ \hline
\texttt{K\_SYSREQ}       &  \verb+ +     &  Systemabfrage\\ \hline
\texttt{K\_BREAK}        &  \verb+ +     &  Abbruch/Unterbrechung\\ \hline
\texttt{K\_MENU}         &  \verb+ +     &  Menü\\ \hline
\texttt{K\_POWER}        &  \verb+ +     &  Ein-/Ausschalten\\ \hline
\texttt{K\_EURO}         &  \verb+€+     &  Euro-Währungszeichen\\ \hline
\texttt{K\_AC\_BACK}     &  \verb+ +     &  Android Zurückschalter\\ \hline
\end{longtable} 

\begin{longtable}{ll}
	\caption{Liste von vordefinierten Konstanten zur Tastaturschaltung}\label{tabKeyMod}\index{Tastatur!-schalter} \\
	% Definition des Tabellenkopfes auf der ersten Seite
	Konstante  & Beschreibung \\\hline\hline
	\hline
	\endfirsthead % Erster Kopf zu Ende
	% Definition des Tabellenkopfes auf den folgenden Seiten
	\caption{Liste von vordefinierten Konstanten zur Tastaturschaltung (Fortsetzung)}\\
	Konstante & Beschreibung \\\hline\hline
	\hline
	\endhead % Zweiter Kopf ist zu Ende
	% Ab hier kommt der Inhalt der Tabelle
    \texttt{KMOD\_NONE}   &  Keine Belegungstaste gedrückt\\ \hline
    \texttt{KMOD\_LSHIFT} &  Linke Umschalttaste\\ \hline
    \texttt{KMOD\_RSHIFT} &  Rechte Umschalttaste\\ \hline
    \texttt{KMOD\_SHIFT}  &  Linke oder rechte Umschalttaste oder beide\\ \hline
    \texttt{KMOD\_LCTRL}  &  Linke Steuerungstaste\\ \hline
    \texttt{KMOD\_RCTRL}  &  Rechte Steuerungstaste\\ \hline
    \texttt{KMOD\_CTRL}   &  Linke oder rechte Steuerungstaste oder beide\\ \hline
    \texttt{KMOD\_LALT}   &  Linke Alternativtaste\\ \hline
    \texttt{KMOD\_RALT}   &  Rechte Alternativtaste\\ \hline
    \texttt{KMOD\_ALT}    &  Linke oder rechte Alternativtaste oder beide\\ \hline
    \texttt{KMOD\_LMETA}  &  Linke Metataste\\ \hline
    \texttt{KMOD\_RMETA}  &  Rechte Metataste\\ \hline
    \texttt{KMOD\_META}   &  Linke oder rechte Metataste oder beide\\ \hline
    \texttt{KMOD\_CAPS}   &  Umschalten Großbuchstaben\\ \hline
    \texttt{KMOD\_NUM}    &  Umschalten Zahlen\\ \hline
    \texttt{KMOD\_MODE}   &  AltGr Umschalter\\ \hline
\end{longtable} 

