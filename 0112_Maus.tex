\newpage
%%%%%%%%%%%%%%%%%%%%%%%%%%%%%%%%%%%%%%%%%%%%%%%%%%%%%%%%%%%%%%%%%%%%%%%%%%%
\section{Maus}\index{Maus}\label{secMaus}

Wie wohl viele Spiele durch Tastatur oder Controler gesteuert werden, wird auch oft die Maus verwendet. In diesem Skript werden die elementaren Mausaktionen wie \emph{Klick} oder \emph{Positionsabfrage} behandelt. Unser Beispiel bildet folgende Funktionalitäten ab:
\begin{itemize}
	\item In der Mitte erscheint eine kleine transparente Blase.
	\item Bewegt sich die Maus innerhalb eines inneren Rechtecks, fungiert die Blase als Mauszeiger.
	\item Verlässt die Maus das innere Rechteck, erscheint der übliche Systemmauszeiger.
	\item Ein Linksklick lässt die Blase um $90°$ nach links rotieren.
	\item Ein Rechtsklick lässt die Blase um $90°$ nach rechts rotieren.
	\item Über das Mausrad wird die Größe der Blase skaliert.
	\item Ein Klick mit dem Mausrad beendet die Anwendung.
\end{itemize}

\myebild{maus00.png}{0.5}{Mausaktionen}{picMaus00}

Die eigentliche Musik spielt in der Hauptklasse \texttt{Game}, da dort die Mausaktionen abgefragt werden. Anstelle einer Klasse \texttt{Settings}, habe ich die Einstellungen hier als statische Variablen und Methoden der Klasse \texttt{Game} implementiert -- das geht auch. Im Konstruktor werden die üblichen Verdächtigen aufgerufen und in \zeiref{srcMaus0001} wird das \texttt{Ball}-Objekt erzeugt.

\newpage

\lstsource{SRC/00 Einführung/12 Maus/maus00.py}{55}{71}{python}{Mausaktionen: Statics und Konstruktor von \texttt{Game}}{srcMaus00a} 

Auch die Methode \texttt{run()} birgt keine Überraschungen.

\lstsource{SRC/00 Einführung/12 Maus/maus00.py}{73}{83}{python}{Mausaktionen: \texttt{Game.run()}}{srcMaus00b} 

In \texttt{watch\_for\_events()} kommen uns die ersten interessanten Stellen unter. Wie bei den Tasten ein \texttt{KEYUP} und ein \texttt{KEYDOWN} das Drücken und Loslassen markieren, gibt es auch Entsprechungen bei der Maus:\ \randnotiz{MOUSEBUTTONDOWN} \randnotiz{MOUSEBUTTONUP}\texttt{MOUSEBUTTONDOWN}\myindex{pyg}{\texttt{MOUSEBUTTONDOWN}|underline} und \texttt{MOUSEBUTTONUP}\myindex{pyg}{\texttt{MOUSEBUTTONUP}|underline}. In \zeiref{srcMaus0002} wird der \texttt{event.type} abgefragt und anschließend wird ermittelt, welche Maustaste denn gedrückt wurde. 

Dazu liefern mir diese beiden Mausevents zwei Attribute: \texttt{event.button} und \texttt{event.pos}. In \tabref[vref]{tabMousebutton} sind die Zahlenkodes von \texttt{event.button}\myindex{pyg}{\texttt{event}!\texttt{button}|underline}\randnotiz{event.button} abgebildet. Erstaunlicherweise gibt es hier keine vordefinierten Konstanten wie bei der Tastatur. Nach der Abfrage werden die entsprechenden Nachrichten an das \texttt{Ball}-Objekt versendet.

Wird also die linke Maustaste gedrückt (\zeiref{srcMaus0004}), wird an den Ball die Nachricht gesendet, sich um $90°$ nach links zu drehen und bei der rechten um $90°$ nach rechts (daher $-90$, siehe \zeiref{srcMaus0006}). Das \index{Mausrad}Mausrad wird ebenfalls wie ein Mausbutton verarbeitet. Je nach Drehrichtung wird dabei ein anderer Zahlenkode zurückgeliefert (siehe \zeiref{srcMaus0007} und \zeiref{srcMaus0008}). Wird das Mausrad gedrückt -- also geklickt -- soll ja das Spiel beendet werden. In \zeiref{srcMaus0005} wird dies abgefragt und umgesetzt.

Mit \texttt{event.pos}\myindex{pyg}{\texttt{event}!\texttt{pos}} könnte man jetzt sofort die Mausposition abfragen -- was wir hier nicht tun.\randnotiz{event.pos}

\lstsource{SRC/00 Einführung/12 Maus/maus00.py}{85}{102}{python}{Mausaktionen: \texttt{Game.watch\_for\_events()}}{srcMaus00c} 

Eine Anforderung war, dass der Systemmauszeiger nur außerhalb des inneren Rechtecks sichtbar ist. Innerhalb des Rechtecks soll ja der Ball als Mauszeiger herhalten. In \zeiref{srcMaus0003} wird dies durch die Methode \texttt{\texttt{pygame.mouse.set\_visible()}}\randnotiz{set\_visible()}\myindex{pyg}{\texttt{mouse}!\texttt{set\_visible()}|underline} erreicht. Diese steuert, ob der Systemmauszeiger -- welcher Ausprägung auch immer -- angezeigt werden soll oder nicht. 

Als Entscheider dient dabei, ob die aktuelle Mausposition innerhalb des inneren Rechtecks liegt. Die Methode \texttt{pygame.mouse.get\_pos()}\randnotiz{get\_pos()}\myindex{pyg}{\texttt{mouse}!\texttt{get\_pos()}|underline} liefert mir dazu die aktuelle Mausposition. Diese wird nun einfach in eine schon vorhandene Kollisionsprüfung gesteckt: \texttt{pygame.Rect.collidepoint()}\randnotiz{collidepoint()}\myindex{pyg}{\texttt{Rect}!\texttt{collidepoint()}|underline}. Ist die Mausposition innerhalb des Rechtecks, liefert diese den Wert \true, ansonsten \false; daher muss der Wahrheitswert noch mit \texttt{not} negiert werden.

\lstsource{SRC/00 Einführung/12 Maus/maus00.py}{104}{115}{python}{Mausaktionen: \texttt{Game.update()} und \texttt{Game.draw()}}{srcMaus00d} 

Verbleibt noch die Klasse \texttt{Ball}. Diese enthält zwar keine direkten Mausaktionen mehr, aber die Methode \texttt{update()} sieht nun ganz anders als bei den vorherigen Beispielen aus. In früheren Beispielen wurden Methoden wie \texttt{rotate()} oder \texttt{resize()} direkt aus \texttt{watch\_for\_events()} oder vergleichbaren Methoden von \texttt{Game} aufgerufen. Das ist auch soweit in Ordnung, aber wenn man diese Kindklassen von \texttt{pygame.sprite.Sprite} einer \texttt{pygame.sprite.Group} oder \texttt{pygame.sprite.GroupSingle} hinzugefügt hat, kriegt man ein Problem. Diese Klassen erwarten nur \texttt{Sprite}-Objekt als Elemente. Deshalb kann man eigentlich im Sinne der objektorientierten Programmierung nur Methoden und Attribute verwwenden, die der Elternklasse \texttt{pygame.sprite.Sprite} bekannt sind -- also beispielsweise \texttt{update()}. Methoden wie \texttt{rotate()} wären dann der Spritegruppe unbekannt.

Nehmen Sie beispielsweise Zeile~82 in \srcref[vref]{srcInvader06d}. Die Methode \texttt{change\_\-direction()} ist dem \texttt{GroupSingle}-Objekt \texttt{defender} völlig unbekannt, da es ein \texttt{Sprite} und kein \texttt{Defender}-Objekt erwartet. Syntax-Checker wie \gls{pylance} werfen hier Fehlermeldungen raus. Eine Möglichkeit das Problem zu umgehen, ist die Verwendung von \texttt{update()} als Verteilstation. In der Klasse \texttt{pygame.sprite.Sprite} wird diese Methode mit folgender Signatur definiert: 

\verb+update(self, *args: Any, **kwargs: Any) -> None+

Mit anderen Worten, man kann der Methode beliebige frei definierbare Parameter übergeben. Genau das passiert in unserer \texttt{update()}-Methode. Bei der Rotation wird der Übergabeparameter \texttt{rotate} mit einem entsprechenden Winkel übergeben, bei der Skalierung der Parameter \texttt{scale} und in \texttt{update()} von \texttt{Game} der Parameter \texttt{go} mit dem Wert \true. Jeder Aufrufer kann also seine Übergabeparameter spontan definieren und mit Werten versehen. Der \texttt{update()} in der Kindklasse -- hier \texttt{Ball} -- muss dies nur abfragen. 

Dabei wird im ersten Schritt gefragt, ob der Parameter übergeben wurde wie in \zeiref{srcMaus0009}, \zeiref{srcMaus0010}, \zeiref{srcMaus0011} oder \zeiref{srcMaus0012}. Anschließend wird der Parameterwert der entsprechenden Methode der Kindklasse übergeben. Somit muss die Spritegruppe nicht auf Methoden der Kindklasse zugreifen, sondern kann die Methode der Elternklasse verwenden.


\lstsource{SRC/00 Einführung/12 Maus/maus00.py}{9}{52}{python}{Mausaktionen: \texttt{Ball}}{srcMaus00e} 

Noch ein Hinweis zu \texttt{pygame.transform.rotate()}\randnotiz{rotate()}\myindex{pyg}{\texttt{transform}!\texttt{rotate()}|underline}. Anders als bei vielen anderen Systemen, die Winkel verarbeiten, wird der Winkel hier in \gls{grad} und nicht in \gls{radiant} gemessen.


\subsection*{Was war neu?}

Mausaktionen werden ähnlich wie Tastaturevents verarbeitet. Die Mausposition kann einfach abgefragt werden. Es ist einfacher den Mauszeiger unsichtbar zu setzen und ein Bitmap der Mausposition folgen zu lassen, als einen neuen Mauszeiger zu setzen.

Es wurden folgende Pygame-Elemente eingeführt:
\begin{itemize}
	\item \texttt{pygame.mouse.get\_pos()}:
    \myindex{pyg}{\texttt{mouse}!\texttt{get\_pos()}}\\ \url{https://www.pygame.org/docs/ref/mouse.html#pygame.mouse.get_pos}

	\item \texttt{pygame.mouse.set\_visible()}:
    \myindex{pyg}{\texttt{mouse}!\texttt{set\_visible()}}\\ \url{https://www.pygame.org/docs/ref/mouse.html#pygame.mouse.set_visible}

	\item \texttt{pygame.MOUSEBUTTONDOWN}, \texttt{pygame.MOUSEBUTTONDOWN}:
    \myindex{pyg}{\texttt{MOUSEBUTTONDOWN}}\myindex{pyg}{\texttt{MOUSEBUTTONUP}}\\ \url{https://www.pygame.org/docs/ref/event.html}

	\item Liste der Mausbuttons: \tabref[vref]{tabMousebutton}

	\item \texttt{pygame.Rect.collidepoint()}:
    \myindex{pyg}{\texttt{Rect}!\texttt{collidepoint()}}\\ \url{https://www.pygame.org/docs/ref/rect.html#pygame.Rect.collidepoint}

	\item \texttt{pygame.transform.rotate()}:
    \myindex{pyg}{\texttt{transform}!\texttt{rotate()}}\\ \url{https://www.pygame.org/docs/ref/transform.html#pygame.transform.rotate}

\end{itemize}


\begin{longtable}{ll}
	\caption{Liste der Mausbuttons}\label{tabMousebutton}\myindex{pyg}{\texttt{event}!\texttt{button}}  \\
	% Definition des Tabellenkopfes auf der ersten Seite
	Konstante & Beschreibung \\\hline\hline
	\hline
	\endfirsthead % Erster Kopf zu Ende
	% Definition des Tabellenkopfes auf den folgenden Seiten
	\caption{Liste der Mausbuttons (Fortsetzung)}\\
	Konstante & Beschreibung \\\hline\hline
	\hline
	\endhead % Zweiter Kopf ist zu Ende
	% Ab hier kommt der Inhalt der Tabelle
	\texttt{0} &  nicht definiert \\ \hline
	\texttt{1} &  linke Maustaste\\ \hline
	\texttt{2} &  mittlere Maustaste/Mausrad\\ \hline
	\texttt{3} &  rechte Maustaste\\ \hline
	\texttt{4} &  Mausrad zu sich drehen (up)\\ \hline
	\texttt{5} &  Mausrad von sich weg drehen (down)\\ \hline
\end{longtable} 
